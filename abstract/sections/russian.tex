Многие задачи составления расписаний сводятся к реберным раскраскам графов с различными ограничениями. В частности, задачи существования и построения расписаний без ``окон'' моделируются с помощью интервальных реберных раскрасок графов и мультиграфов. Настоящая диссертация посвящена исследованию интервальных реберных раскрасок и их обобщений.

В диссертационной работе рассматриваются конечные неориентированные графы без кратныx ребер и петель, а также мультиграфы, которые допускают наличие кратных ребер. Пусть $V(G)$ --- множество вершин мультиграфа $G$, а $E(G)$ --- множество ребер мультиграфа $G$. Функция $\alpha : E(G) \rightarrow \mathbb{N}$ называется правильной реберной раскраской мультиграфа $G$, если смежные ребра окрашены в различные цвета. Если $\alpha$ --- правильная реберная раскраска мультиграфа $G$ и $v\in V(G)$, то через $S(v,\alpha)$ обозначим множество цветов ребер, инцидентных $v$. Обозначим через $\underline{S}(v, \alpha)$ и $\overline{S}(v, \alpha)$ наименьший и наибольший цвет из множества $S(v,\alpha)$, соответственно.

Правильная реберная раскраска $\alpha : E(G) \rightarrow \{1,\ldots,t\}$ называется интервальной $t$-раскраской мультиграфа $G$, если каждый из $t$ цветов использован и для любой вершины $v\in V(G)$ множество $S(v,\alpha)$ образует интервал целых чисел. Мультиграф $G$ называется интервально раскрашиваемым, если он обладает интервальной $t$-раскраской для некоторого $t$. Обозначим через $\mathfrak{N}$ множество интервально раскрашиваемых мультиграфов. Для мультиграфа $G \in \mathfrak{N}$ через $w(G)$ и $W(G)$ обозначим, соответственно, наименьшее и наибольшее $t$, при котором $G$ обладает интервальной $t$-раскраской.

Пусть $G$ --- произвольный мультиграф. Для любой правильной реберной раскраски $\alpha$ мультиграфа $G$ определим дефицит раскраски  $\alpha$ следующим образом: $def(G, \alpha) = \sum\limits_{v\in V(G)}{\left(\overline{S}(v,\alpha)-\underline{S}(v,\alpha) - |S(v,\alpha)|+1\right)}$. Дефицитом мультиграфа $G$ называется число $def(G)$, определяемое следующим образом: $def(G) = \min_{\alpha}{def(G,\alpha)}$, где минимум берется по всевозможным правильным реберным раскраскам $G$. Для мультиграфа $G$ через $w_{def}(G)$ и $W_{def}(G)$ обозначим, соответственно, наименьшее и наибольшее $t$, при котором $G$ обладает правильной реберной раскраской $\alpha$ в $t$ цветов с дефицитом $def(G, \alpha) = def(G)$.

Декартовым произведением графов $G$ и $H$ называется граф $G \square H$, определяемый следующим образом:
\begin{align*}
V(G \square H) &= V(G) \times V(H) \\
E(G \square H) &= \left\{(u_1,v_1)(u_2,v_2) : (u_1=u_2 \text{ и } v_1v_2 \in E(H)) \text{ или } (v_1=v_2 \text{ и } u_1u_2 \in E(G) \right\}
\end{align*}

В настоящей работе исследованы задачи существования, построения и оценки числовыx параметров интервальныx реберных раскрасок различных классов мультиграфов, а также задачи устойчивости интервальных реберных раскрасок относительно некоторых графовых операций. Кроме того, особое внимание в работе уделено задачам определения и оценивания числовых параметров $def(G)$, $w_{def}(G)$ и $W_{def}(G)$ для интервально не раскрашиваемых графов. В частности, получены следующие результаты: 

\begin{enumerate}
    \item общие оценки параметров $w(G)$ и $W(G)$ ($w_{def}(G)$ и $W_{def}(G)$) интервально раскрашиваемых (не раскрашиваемых) мультиграфов;

    \item новые нижние и верхние оценки параметра $W(K_{2n})$, основанные на эквивалентности интервальных реберных раскрасок полного графа $K_{2n}$ и специального типа факторизации этого графа, а также точные значения $W(K_{2n})$ при $n\leq 12$;
    
    \item новые результаты, касающиеся интервальной раскрашиваемости некоторых классов графов, а также новые оценки параметров $w(G)$ и $W(G)$ для полных многодольных графов, внешнепланарных графов, сеток, цилиндров, торов и графов Хэмминга (в некоторых случаях точные значения этих параметров);
    
    \item возможные связи между интервальной раскрашиваемостью декартовых произведений графов и их факторов, существенно улучшенные оценки параметра $W(G \square H)$ для регулярных графов, а также интервальная раскрашиваемость некоторых классов декартовых произведений  двудольных графов;
    
    \item дефициты некоторых графов, подтверждение гипотезы Боровецка-Олшевской, Дргаш-Буршардт и Халушчака, а также справедливость гипотезы о дефиците графа ($def(G) \leq |V(G)|$ для произвольного графа $G$) в случае внешнепланарных графов;
    
    \item частичное решение проблемы Дженсена и Тофта о наименьшем интервально не раскрашиваемом двудольном графе, построение наименьших известных таких графов, а также интервальная раскрашиваемость всех двудольных графов с числом вершин не превосходящим $16$.

\end{enumerate}