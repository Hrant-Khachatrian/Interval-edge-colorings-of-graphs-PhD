Երկրորդ գլուխն ավարտենք Հեմմինգի գրաֆների միջակայքային ներկումների ուսումնասիրությամբ: $k$ լրիվ գրաֆների դեկարտյան արտադրյալը կոչվում է Հեմմինգի գրաֆ.
\begin{center}
$ H(n_1,n_2,\ldots,n_k) = K_{n_1} \square K_{n_2} \square \ldots \square K_{n_k}$,\\
$ H_n^k = \underset{k}{\underbrace{K_n \square \ldots \square K_n}}$:
\end{center}

Պետրոսյանը \cite{Petrosyan2011} ցույց է տվել, որ Հեմմինգի գրաֆները ունեն միջակայքային ներկում այն և միայն այն դեպքում, երբ բաղադրիչներից գոնե մեկը զույգ գագաթանի լրիվ գրաֆ է: Թեորեմ \ref{t1_regular}-ից՝ երբ $H(n_1,\ldots,n_k)\in \mathfrak{N}$
\begin{center}
$w(H(n_1,\ldots,n_k))=\Delta(H(n_1,\ldots,n_k))=\sum\limits_{i=1}^k{n_i}-k$:
\end{center}

Այն դեպքի համար, երբ Հեմմինգի գրաֆի բոլոր բաղադրիչները $2n$ գագաթներով լրիվ գրաֆներ են, Պետրոսյանը \cite{Petrosyan2011} ստացել էր այսպիսի գնահատական.
\begin{theorem}
\label{t2_Hamming_balanced_lower_old} Երբ $n=p2^q$, որտեղ $p$-ն կենտ է, $q\geq0$,
\begin{center}
$W(H_{2n}^k) \geq (4n-2-p-q)k$:
\end{center}
\end{theorem}

Օգտվելով Թեորեմ \ref{t2_regular_k_product}-ից, Հեմմինգի գրաֆի միջակայքային ներկումներում մասնակցող գույների առավելագույն թիվը կարելի է գնահատել լրիվ գրաֆների միջակայքային ներկումներում մասնակցող գույների թվերով, երբ Հեմմինգի գրաֆի բոլոր բաղադրիչները ունեն զույգ թվով գագաթներ:
\begin{theorem}
\label{t2_Hamming_balanced_lower} Եթե $G = H(2n_1,\ldots,2n_k)$, որտեղ $n_1,\ldots,n_k \in \mathbb{N}$, ապա
\begin{center}
$W(G) \geq \sum\limits_{i=1}^k{W(K_{2n_i})}+\sum\limits_{i=1}^{k-1}{i\left(2n_i-1\right)}$:
\end{center}
\end{theorem}
Մասնավորապես, երբ Հեմմինգի գրաֆների բոլոր բաղադրիչները իրար հավասար են, Թեորեմ \ref{t1-complete-W-lower-best}-ը հնարավորություն է տալիս ստանալ այսպիսի գնահատական.

\begin{corollary}
\label{c2_Hamming_balanced_lower} Եթե $n = \prod\limits_{i=1}^{\pi(n)}{p_i^{\alpha_i}}$, որտեղ $p_i$-ն $i$-րդ պարզ թիվն է, $\pi(n)$-ը՝ $n$-ը չգերազանցող պարզ թվերի քանակը, իսկ $\alpha_i \in \mathbb{Z}_{\geq 0}$, ապա
\begin{center}

$W(H_{2n}^k) \geq (4n - 3 - A_n)k + \frac{1}{2}k(k-1)(2n-1)$,
\end{center}
որտեղ $A_n = \alpha_1 + 2\alpha_2 + 3\alpha_3 + 4\alpha_4 + 4\alpha_5 + \frac{1}{2}\sum\limits_{i=6}^{\pi(n)}{\alpha_i(p_i+1)} $:
\end{corollary}
Նկատենք, որ այս արդյունքը էապես լավացնում է Թեորեմ \ref{t2_Hamming_balanced_lower_old}-ի գնահատականը: Այս հետևանքից և Թեորեմ \ref{t1_regular}-ից ստանում ենք հետևյալ պնդումը.
\begin{corollary}
Եթե $n = \prod\limits_{i=1}^{\infty}{p_i^{\alpha_i}}$, որտեղ $p_i$-ն $i$-րդ պարզ թիվն է, $\alpha_i \in \mathbb{Z}_{\geq 0}$, իսկ $k(2n-1) \leq t \leq (4n - 3 - A_n)k + \frac{1}{2}k(k-1)(2n-1)$,
որտեղ $A_n = \alpha_1 + 2\alpha_2 + 3\alpha_3 + 4\alpha_4 + 4\alpha_5 + \frac{1}{2}\sum\limits_{i=6}^{\infty}{\alpha_i(p_i+1)}$, ապա $H_{2n}^k$ գրաֆը ունի միջակայքային $t$-ներկում:
\end{corollary}

Այժմ Հեմմինգի գրաֆի միջակայքային ներկումներում մասնակցող գույների առավելագույն թիվը գնահատենք վերևից:

\begin{theorem}
\label{t2_Hamming_upper}
Եթե $G = H(2n_1,\ldots,2n_k)$, որտեղ $n_1, n_2, \ldots, n_k \in \mathbb{N}$, ապա
\begin{center}
$W(G) \leq \frac{1}{2}(k+1)\sum\limits_{i=1}^{k}{(4n_i-3)}$:
\end{center}
\end{theorem}

\begin{proof}[Ապացույց]
Դիտարկենք $G=H(2n_1,\ldots,2n_k)$ գրաֆի որևէ $\alpha$ միջակայքային $W(G)$-ներկում: Կամայական երկու $e$ և $e'$ կողերի համար նշանակենք $sp_{\alpha}(e,e') = |\alpha(e) - \alpha(e')|$: Նշանակենք $sp_{\alpha,m} = \max\limits_{d(e,e')=m}{sp_{\alpha}(e,e')}$: Փորձենք գնահատել այս մեծությունը: Պարզ է, որ $sp_{\alpha,0} = \Delta(G)-1$: 

Ենթադրենք, որ $m \geq 1$, և ֆիքսենք $e$ և $e'$ կողեր, որոնց հեռավորությունը $m$ է: Գոյություն ունեն $u$ և $v$ գագաթներ, որոնք հանդիսանում են, համապատասխանաբար, $e$-ի և $e'$-ի ծայրակետ, և որոնց հեռավորությունը $m$ է: Առանց ընդհանրությունը խախտելու կարող ենք համարել, որ $\alpha(e) \geq \alpha(e')$: Հեմմինգի գրաֆի հատկություններից հետևում է, որ գոյություն ունեն $u$ և $v$ գագաթները միացնող գագաթներով չհատվող $m$ շղթաներ: Հետևաբար գոյություն ունեն $v_1, \ldots, v_m$ գագաթներ այնպես, որ $vv_i \in E(G)$ և $d(v_i,u)=m-1$, $i=1,\ldots,m$:  Այդ կողերից ամենափոքր գույնով ներկված կողը նշանակենք $e''$-ով: Պարզ է, որ $\alpha(e'') \leq \alpha(e') + \Delta(G) - m$: Հետևաբար՝
\begin{align*}
sp_{\alpha}(e, e') &= |\alpha(e) - \alpha(e')| \leq |\alpha(e) - \alpha(e'') + \Delta(G) - m| \leq \\
& \leq |\alpha(e)-\alpha(e'')| + \Delta(G) -m \leq sp_{\alpha,m-1} + \Delta(G) - m:
\end{align*}
Քանի որ $e$ և $e'$ կողերի ընտրությունը պատահական էր, կարող ենք պնդել, որ $sp_{\alpha,m} \leq sp_{\alpha,m-1} + \Delta(G) - m$: Կատարելով մաթեմատիկական ինդուկցիա ըստ $m$-ի՝ կստանանք.
\begin{center}
$sp_{\alpha,m} \leq (m+1)\Delta(G)-\frac{1}{2}m(m+1) - 1$:
\end{center}

Նկատենք, որ $G=H(2n_1, \ldots, 2n_k)$ գրաֆում երկու կողերի հեռավորությունը չի կարող գերազացնել $k$-ն: Դիցուք $e_1,e_W \in E(G)$ կողերը ներկված են, համապատասխանաբար, $1$ և $W(G)$ գույներով, իսկ $d(e_1,e_W)=m_0 \leq k$: Ունենք, որ
\begin{align*}
W(G)-1 &= sp_{\alpha}(e_1,e_W) \leq (m_0+1)\left(\Delta(G) - \frac{m_0}{2}\right) - 1 \leq (k+1)\left(\Delta(G) - \frac{k}{2}\right) - 1,\\
W(G) &\leq (k+1)\left(\sum\limits_{i=1}^{k}{(2n_i-1)} - \frac{k}{2}\right) = \frac{1}{2}(k+1)\sum\limits_{i=1}^{k}{(4n_i-3)}:
\end{align*}
\end{proof}


\begin{corollary}
\label{c2_Hamming_balanced_upper} Եթե $n \in \mathbb{N}$, ապա 
$W(H_{2n}^k) \leq \frac{1}{2}k(k+1)(4n-3)$:
\end{corollary}

Հեմմինգի գրաֆների մասնավոր դեպքն են հանդիսանում $n$-չափանի խորանարդները՝ $Q_n = H_2^n$: Այս գրաֆների համար Հետևանքներ \ref{c2_Hamming_balanced_lower}-ում և \ref{c2_Hamming_balanced_upper}-ում ստացված ստորին և վերին գնահատականները համընկնում են և տալիս են $W(Q_n)$-ի ճշգրիտ արժեքը.

\begin{corollary}
\label{c2_n_cube} Եթե $n\in\mathbb{N}$, ապա $W\left(Q_{n}\right) = \frac{n(n+1)}{2}$:
\end{corollary}

Այս արժեքը որպես ստորին գնահատական առաջին անգամ ստացվել էր \cite{Petrosyan2010}-ում, իսկ որպես ճշգրիտ արժեք՝ \cite{PetrosyanKhachatrianTananyan2011,PetrosyanKhachatrianTananyan2013}-ում:
