\bibitem{AltinakarCaporossiHertz} H.S. Altinakar, G. Caporossi, A. Hertz, A comparison of integer and constraint programming models for the deficiency problem, Computers and Oper. Res. 68, 2016, pp. 89-96.
\bibitem{AppelHaken1} K. Appel, W. Haken, Every planar map is four colorable, Part I, Discharging, Illinois Journal of Mathematics 21, 1977, pp. 429-490.
\bibitem{AppelHaken2} K. Appel, W. Haken, J. Koch, Every planar map is four colorable, Part II, Reducibility, Illinois Journal of Mathematics 21, 1977, pp. 491-567.
\bibitem{Asratian2000} A.S. Asratian, Some results on an edge coloring problem of Folkman and Fulkerson, Discrete Math. 223, 2000, pp. 13-25.
\bibitem{AsratianKamalian1994} A.S. Asratian, R.R. Kamalian, Investigation on interval edge-colorings of graphs, J. Combin. Theory Ser. B 62, 1994, pp. 34-43. 
\bibitem{AsratianDenleyHaggvist1998} A.S. Asratian., T.M.J. Denley, R. Haggkvist, Bipartite graphs and their applications, Cambridge Tracts in Mathematics, Cambridge University Press, 1998.
\bibitem{AsratianCasselgrenPetrosyan2017} A.S. Asratian, C.J. Casselgren, P.A. Petrosyan, Some results on cyclic interval colorings of graphs, J. Graph Theory, 2017, ընդունված է տպագրության:
\bibitem{AsratianCasselgrenVandenbusscheWest2009} A.S. Asratian, C.J. Casselgren, J. Vandenbussche, D.B. West, Proper path-factors and interval edge-coloring of (3,4)-biregular bigraphs, J. Graph Theory 61, 2009, pp. 88-97.
\bibitem{Axenovich2002} M.A. Axenovich, On interval colorings of planar graphs, Congressus Numerantium 159, 2002, pp. 77-94.
\bibitem{BehzadMahmoodian1969} M. Behzad, E.S. Mahmoodian, On topological invariants of the product of graphs, Canad. Math. Bull., 12, 1969, pp. 157-166.
\bibitem{BeinekeWilson} L.W. Beineke, R.J. Wilson, On the edge-chromatic number of a graph, Discrete Math. 5, 1973, pp. 15-20.
\bibitem{Berge1958} C. Berge, Theorie des Graphes et ses Applications, Dunod, Paris, 1958.
\bibitem{BorowieckaDrgas2016} M. Borowiecka-Olszewska, E. Drgas-Burchardt, The deficiency of all generalized Hertz graphs and minimal consecutively non-colourable graphs in this class. Discrete Math. 339, 2016, pp. 1892-1908.
\bibitem{B-OD-BHal} M. Borowiecka-Olszewska, E. Drgas-Burchardt, M. Ha\l uszczak, On the structure and deficiency of $k$-trees with bounded degree, Discrete Appl. Math. 201, 2016, pp. 24-37.
\bibitem{BouchardHertzDesaulniers} M. Bouchard, A. Hertz, G. Desaulniers, Lower bounds and a tabu search algorithm for the minimum deficiency problem, J. Comb. Optim. 17, 2009, pp. 168-191.
\bibitem{CasselgrenPetrosyanToft2017} C.J. Casselgren, P.A. Petrosyan, B. Toft, On interval and cyclic interval edge colorings of (3,5)-biregular graphs, Discrete Mathematics, 2017, ընդունված է տպագրության:
\bibitem{CarlJToft} C.J. Casselgren, B. Toft, On interval edge colorings of biregular bipartite graphs with small vertex degrees, J. Graph Theory 80, 2015, pp. 83-97.% http://dx.doi.org/10.1002/jgt.21841
\bibitem{CrowdProcess} CrowdProcess distributed computing platform [առցանց].  \url{http://www.crowdprocess.com/}
\bibitem{DeWerra1971Balanced} D. de Werra, Balanced schedules, INFOR. N9, 1971, pp. 230-237.
\bibitem{DeWerra1971} D. de Werra, Investigations on an edge coloring problem, Discrete Math. 1, 1971, pp. 167-179.
\bibitem{DeWerraSolot1991} D. de Werra, Ph. Solot, Compact cylindrical chromatic scheduling, SIAM J. Discrete Math. 4, 1991, pp. 528-534.
\bibitem{DulmageMendelsohn} A.L. Dulmage, N.S. Mendelsohn, Some graphical properties of matrices with nonnegative entries, Acquat. Math., 2, 1969, pp. 150-162.
\bibitem{ErdosRubinTaylor} P. Erdős, A.L. Rubin, H. Taylor, Choosability in graphs, Proc. West Coast Conf. on Combinatorics, Graph Theory and Computing, Congr. Numer. 26, 1979, pp. 125-157.
\bibitem{EvenItaiShamir} S. Even, A. Itai, A. Shamir, On the complexity of timetable and multicommodity flow problems, SIAM J. Comput. 5 (4), 1976, pp. 691-703.
\bibitem{FengHuang2007} Y. Feng, Q. Huang, Consecutive edge-coloring of the generalized $\theta$-graph, Discrete Appl. Math. 155, 2007, pp. 2321-2327. %doi:10.1016/j.dam.2007.06.010
\bibitem{Fiorini1975} S. Fiorini, On the chromatic index of outerplanar graphs, J. Combin. Theory Ser. B 18, 1975, pp. 35-38. %doi:10.1016/0095-8956(75)90060-X
\bibitem{FolkmanFulkerson} J. Folkman, D.R. Fulkerson, Edge colourings in bipartite graphs, in Combinatorial  Mathematics and its Applications, University of North Carolina Press, Chapel Hill, 1969, pp. 561-577.
\bibitem{Giaro1997} K. Giaro, The complexity of consecutive $\Delta $-coloring of bipartite graphs: $4$ is easy, $5$ is hard, Ars Combin. 47, 1997, pp. 287-298.
\bibitem{Giaro1999} K. Giaro, Compact task scheduling on dedicated processors with no waiting periods, PhD thesis, Technical University of Gdansk, EIT faculty, Gdansk, 1999 (լեհերեն).
\bibitem{GiaroKubale1997}	K. Giaro, M. Kubale, Consecutive edge-colorings of complete and incomplete Cartesian products of graphs, Cong, Num. 128, 1997, pp. 143-149. 
\bibitem{GiaroKubale2004}	K. Giaro, M. Kubale, Compact scheduling of zero-one time operations in multi-stage systems, Discrete Appl. Math. 145, 2004, pp. 95-103.
\bibitem{GiaroKubaleMalafiejski1999} K. Giaro, M. Kubale, M. Malafiejski, On the deficiency of bipartite graphs, Discrete Appl. Math. 94, 1999, pp. 193-203.
\bibitem{GiaroKubaleMalafiejski2001} K. Giaro, M. Kubale, M. Malafiejski, Consecutive colorings of the edges of general graphs, Discrete Math. 236, 2001, pp. 131-143.
\bibitem{HansonLoten} D. Hanson, C.O.M. Loten, A lower bound for interval colouring bi-regular bipartite graphs, Bulletin of the ICA 18, 1996, pp. 69-74.
\bibitem{HansonLotenToft1998}	D. Hanson, C.O.M. Loten, B. Toft, On interval colorings of bi-regular bipartite graphs, Ars Combin. 50, 1998, pp. 23-32.
\bibitem{Harary1969} F. Harary, Graph Theory, Addison-Wesley, Reading, MA, 1969. 
\bibitem{Harary1974} H.J. Fleischner, D.P. Geller, F. Harary, Outerplanar graphs and weak duals, J. Indian Math. Soc. 38, 1974, pp. 215-219.
\bibitem{Hansen1992} H.M. Hansen, Scheduling with minimum waiting periods, Master's Thesis, Odense University, Odense, Denmark, 1992 (դանիերեն).
\bibitem{HudakEtAl2016} P. Hudák, F. Kardoš, T. Madaras, M. Vrbjarová, On improper interval edge colourings, Czechoslovak Mathematical Journal 66 (4), 2016, pp. 1119-1128.
\bibitem{JensenToft1995} T.R. Jensen, B. Toft, Graph coloring problems, Wiley Interscience Series in Discrete Mathematics and Optimization, 1995.
\bibitem{KamalianPetrosyan2012}	R.R. Kamalian, P.A. Petrosyan, A note on upper bounds for the maximum span in interval edge-colorings of graphs, Discrete Math. 312, 2012, pp. 1393-1399.
\bibitem{Konig1916} D. König, \textit{Uber Graphen und ihre Anwendung auf Determinantentheorie und Mengenlehre}, Math. Ann. 77, 1916, pp. 453-465.
\bibitem{Kubale2004} M. Kubale, Graph Colorings, American Mathematical Society, 2004.
\bibitem{KubickaSum} E. Kubicka, The chromatic sum of a graph, PhD Thesis, Western Michigan University, 1989.
\bibitem{McKay} B.D. McKay, A. Piperno, Practical Graph Isomorphism, II, J Symbolic Computation, vol. 60, 2014, pp. 94-112. %http://dx.doi.org/10.1016/j.jsc.2013.09.003
\bibitem{Petrosyan2010}	P.A. Petrosyan, Interval edge-colorings of complete graphs and n-dimensional cubes, Discrete Math. 310, 2010, pp. 1580-1587.
\bibitem{Petrosyan2011}	P.A. Petrosyan, Interval edge colorings of some products of graphs, Discuss. Math. Graph Theory 31(2), 2011, pp. 357-373.
\bibitem{Petrosyan2012}	P.A. Petrosyan, Interval colorings of complete balanced multipartite graphs, arxiv:1211.5311, 2012.
\bibitem{Petrosyan2013}	P.A. Petrosyan, On Interval Non-Edge-Colorable Eulerian Multigraphs, arxiv:1311.2210, 2013.
\bibitem{Petrosyan2013Outerplanar} P.A. Petrosyan, On Interval Edge-Colorings of Outerplanar Graphs, Ars Combin. 132, 2017, pp. 127-135.
\bibitem{PetrosyanArakelyan2007} P.A. Petrosyan, H.Z. Arakelyan, On a generalization of interval edge colorings of graphs. Transactions of IPIA of NAS RA, Math. Probl. of Comp. Sci. 29, 2007, pp. 26-32.
\bibitem{PetrosyanKarapetyan2007} P.A. Petrosyan, G.H. Karapetyan, Lower bounds for the greatest possible number of colors in interval edge colorings of bipartite cylinders and bipartite tori, Proceedings of the CSIT Conference, Yerevan, 2007, pp. 86-88.
\bibitem{PetrosyanKhachatrianCID2013} P.A. Petrosyan, H.H. Khachatrian, On a generalization of interval edge-colorings of graphs, 15th Workshop on Graph Theory, Colourings, Independence and Domination, Szklarska Poreba, Poland, 2013, p. 44.
\bibitem{PetrosyanKhachatrian2014} P.A. Petrosyan, H.H. Khachatrian, Interval non-edge-colorable bipartite graphs and multigraphs, J. Graph Theory 76, 2014, pp. 200-216.
\bibitem{PetrosyanKhachatrianTananyan2011}	P.A. Petrosyan, H.H. Khachatrian, H.G. Tananyan, Interval edge-colorings of Cartesian products of graphs, CID abstracts, 2011, p.44. http://www.cid.uz.zgora.pl/2011/files/AbstractsPdf/Khachatrian.pdf
\bibitem{PetrosyanKhachatrianTananyan2013}	P.A. Petrosyan, H.H. Khachatrian, H.G. Tananyan, Interval edge-colorings of Cartesian products of graphs I, Discuss. Math. Graph Theory 33(3), 2013, pp. 613-632.
\bibitem{PetrosyanKhachatrianYepremyanTananyan2011} P.A. Petrosyan, H.H. Khachatrian, L.E. Yepremyan, H.G. Tananyan, Interval edge-colorings of graph products, Proceedings of the CSIT Conference, 2011, pp. 89-92.
\bibitem{PetrosyanMkhitaryan} P.A. Petrosyan, S.T. Mkhitaryan, Interval cyclic edge-colorings of graphs, Discrete Math. 339, 2016, pp. 1848-1860.
\bibitem{Pyatkin2004}	A.V. Pyatkin, Interval coloring of (3,4)-biregular bipartite graphs having large cubic subgraphs, J. Graph Theory 47, 2004, pp. 122-128.
\bibitem{Sabidussi1960} G. Sabidussi, Graph multiplication, Math. Z. 72, 1960, pp. 446-457.
\bibitem{Shannon1949} C.E. Shannon, A theorem on colouring the lines of a network, J. Math. Phys. 28, 1949, pp. 148-151.
\bibitem{Schwartz2006} A. Schwartz, The deficiency of a regular graph. Discrete Math. 306, 2006, pp. 1947-1954.
\bibitem{Stiebitz2012} M. Stiebitz, D. Scheide, B. Toft, L.M. Favrholdt, Graph Edge Coloring: Vizing's Theorem and Goldberg's Conjecture, Wiley Interscience Series in Discrete Mathematics and Optimization, 2012. 
\bibitem{SupowitSum} K.J. Supowit, Finding a maximum planar subset of nets in a channel, IEEE Trans. Comput. Aided Design CAD 6(1), 1987, pp. 93-94.
\bibitem{TepanyanPetrosyan} H.H. Tepanyan, P.A. Petrosyan, Interval edge-colorings of composition of graphs, Discrete Appl. Math. 217, 2017, pp. 368-374.
\bibitem{West1996}	D.B. West, Introduction to Graph Theory, Prentice-Hall, New Jersey, 1996.
\bibitem{YangLi2011}	F. Yang, X. Li, Interval coloring of (3,4)-biregular bigraphs having two (2,3)-biregular bipartite subgraphs, Applied Math. Letters 24, 2011, pp. 1574-1577. 
\bibitem{AsratianKamalian1987} А.С. Асратян, Р.Р. Камалян, Интервальные раскраски ребер мультиграфа, Прикладная математика, вып. 5, 1987, стр. 25-34.
\bibitem{Vizing1963} В.Г. Визинг, Декартовое произведение графов, Вычис. Системы 9, 1963, стр. 30-43. 
\bibitem{Vizing1965} В.Г. Визинг, Хроматический класс мультиграфов, Кибернетика 3, 1965, стр. 29-39.
\bibitem{Vizing1965critical} В.Г. Визинг, Критические графы с данным хроматическим классом, Дискретный анализ 5, 1965, стр. 9-17.
\bibitem{VizingVertex} В.Г. Визинг, Раскраска вершин графа в предписанные цвета, Методы дискретного анализа 29, 1976, стр. 3-10.
\bibitem{Kamalian1989}	Р.Р. Камалян, Интервальные раскраски полных двудольных графов и деревьев, Препринт ВЦ АН Арм. ССР и ЕГУ, Ереван, 1989, 11 стр. 
\bibitem{Kamalian1990}	Р.Р. Камалян, Интервальные реберные раскраски графов, канд. дисс., Новосибирск, 1990.
\bibitem{Kamalian2010}	Р.Р. Камалян, Об одном классе планарных двудольных графов, не имеющих интервальной реберной раскраски, Գիտական հոդվածների ժողովածու, ԵՊՀ Իջևանի մասնաճյուղ, 2010, стр. 149-151
\bibitem{KamalianMirumyan1997}	Р.Р. Камалян, А.Н. Мирумян, Интервальные реберные раскраски двудольных графов одного класса, Доклады НАН РА, том 97, N 4, 1997, стр. 3-5. 
\bibitem{PetrosyanKhachatrian2010} П.А. Петросян, Г.А. Хачатрян, Интервальные реберные раскраски декартовых произведений регулярных графов, Пятая годичная научная конференция РАУ, Сборник научных работ, 2010, стр. 241-248.
\bibitem{Pyatkin2015} А.В. Пяткин, Об интервальной (1,1)-раскраске инциденторов интервально раскрашиваемых графов, Дискретн. анализ и исслед. опер. 22:2, 2015, стр. 63-72.
\bibitem{Sevastyanov1990} С.В. Севастьянов, Об интервальной раскрашиваемости ребер двудольного графа, Методы дискретного анализа в решении экстремальных задач, вып. 50, 1990, стр. 61-72.
\bibitem{Yepremyan2011}	Լ. Եփրեմյան, Գրաֆների արտադրյալների միջակայքային ներկումների մասին, մագիստրոսական թեզ, Երևանի Պետական Համալսարան, 2011, 56 էջ:
\bibitem{Kchoyan2010} Ա. Խչոյան, Ենթախորանարդ գրաֆների և մուլտիգրաֆների միջակայքային կողային ներկումներ, ավարտական աշխատանք, Երևանի Պետական Համալսարան, 2010, 30 էջ:
\bibitem{MamikonyanGithub} Տ. Մամիկոնյան, graphonline բաշխված համակարգ [առցանց]. \url{https://github.com/sarahmiracle/graphonline}
\bibitem{PMK2015} Պ.Ա. Պետրոսյան, Վ.Վ. Մկրտչյան, Ռ.Ռ. Քամալյան, Գրաֆների տեսություն, ուսումն. ձեռն., Երևան, ԵՊՀ հրատ., 2015: