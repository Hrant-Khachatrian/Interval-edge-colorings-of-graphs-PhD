Առաջին գլխի 1.1 պարագրաֆը նվիրված է միջակայքային ներկելիության անհրաժեշտ պայմաններին: Հասրաթյանը և Քամալյանը\footnote{A.S. Asratian, R.R. Kamalian, Investigation on interval edge-colorings of graphs, J. Combin. Theory Ser. B 62, 1994, pp. 34-43.} ցույց են տվել, որ եթե $G$ մուլտիգրաֆը ունի միջակայքային ներկում, ապա $\chi^{\prime}(G)=\Delta(G)$: Երբ $G$-ն համասեռ մուլտիգրաֆ է, այս պայմանը հանդիսանում է նաև բավարար պայման:  Աշխատանքում ստացվել է միջակայքային ներկելիության մեկ այլ անհրաժեշտ պայման:

\begin{hide}
\begin{theorem}
\label{t1_class1} Եթե $G$ մուլտիգրաֆը ունի միջակայքային ներկում, ապա $\chi^{\prime}(G)=\Delta(G)$:
\end{theorem}
\begin{theorem}
\label{t1_regular} Եթե $G$-ն համասեռ մուլտիգրաֆ է, ապա
\begin{description}
\item[(1)] $G\in \mathfrak{N}$ այն և միայն այն դեպքում, երբ $\chi^{\prime}(G)=\Delta(G)$,
\item[(2)] Եթե $G\in \mathfrak{N}$ և $w(G)\leq t\leq W(G)$, ապա $G$-ն ունի միջակայքային $t$-ներկում:
\end{description}
\end{theorem}
\end{hide}

\begin{theorem}
\label{t1_divisor} Եթե $G$ մուլտիգրաֆի համար գոյություն ունի $d$ թիվ, որը $G$-ի բոլոր գագաթների աստիճանների ընդհանուր բաժանարար է, սակայն $\vert E(G)\vert$-ի բաժանարար չէ, ապա $G\notin \mathfrak{N}$:
\end{theorem}
\begin{corollary}
\label{c1_eulerian} Եթե $G$-ն էյլերյան մուլտիգրաֆ է և $\vert
E(G)\vert$ կենտ է, ապա $G\notin \mathfrak{N}$:
\end{corollary}
