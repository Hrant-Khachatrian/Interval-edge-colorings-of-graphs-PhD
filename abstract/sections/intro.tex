\paragraph{Թեմայի արդիականությունը:} 

Դիսկրետ մաթեմատիկայում մեծ ուշադրություն է հատկացվում ներկումների խնդիրների հետազոտություններին: Պատմականորեն ներկումների հանդեպ հետաքրքրությունը պայմանավորված էր «Չորս գույների հիպոթեզ» հանրահայտ խնդրով, համաձայն որի ամեն մի աշխարհագրական քարտեզ հնարավոր է ներկել չորս գույների միջոցով այնպես, որ յուրաքանչյուր երկրի տարածք ներկված լինի մեկ գույնով, իսկ ընդհանուր սահման ունեցող երկրները ներկված լինեն տարբեր գույներով: Դիսկրետ մաթեմատիկայի հետագա զարգացումը ցույց տվեց, որ դա պայմանավորված է ինչպես ներկումների խնդիրների՝ մի շարք կարևոր կիրառական խնդիրների հետ առկա սերտ կապով, այնպես էլ նրանով, որ դիսկրետ մաթեմատիկայում առկա են բազմաթիվ խնդիրներ, որոնք կարելի է ձևակերպել որպես ներկումների խնդիրներ (ֆակտորիզացիայի խնդիրներ, տրոհման խնդիրներ, Ռամսեյի տեսության խնդիրներ և այլն): Մասնավորապես, նշանակալի փոխադարձ կապ կա կարգացուցակների տեսության խնդիրների և գրաֆների ներկումների խնդիրների միջև: Օրինակ, քննաշրջանի օպտիմալ կարգացուցակ կառուցելու խնդիրը բերվում է գրաֆի քրոմատիկ թվի որոշմանը: Գրաֆի քրոմատիկ դասը գտնելու խնդրին բերվում է սպորտային մրցումների կարգացուցակ կազմելու խնդիրը:

Կարգացուցակների տեսության բազմաթիվ խնդիրներ կարելի է բերել ոչ միայն գրաֆների դասական ներկումների խնդիրներին, այլև լրացուցիչ պայմաններով ճիշտ գագաթային և կողային ներկումների գոյության ու կառուցման խնդիրներին: Օրինակ, Ջ. Ֆոլկմանը և Դ. Ֆալկերսոնը\footnote{J. Folkman, D.R. Fulkerson, Edge colourings in bipartite graphs, in Combinatorial  Mathematics and its Applications, University of North Carolina Press, Chapel Hill, 1969, pp. 561-577.} դիտարկել են երկկողմանի մուլտիգրաֆը $r$ գույներով ճիշտ կողային ներկման գոյության խնդիրը, երբ $i$ գույնով ներկված կողերի քանակը $n_i$ է, $i=1,\ldots,r$: Այս խնդիրը համապատասխանում է $r$ ժամ ընդհանուր տևողություն ունեցող այնպիսի ուսումնական դասացուցակի կառուցմանը, երբ $i$-րդ ժամի ընթացքում զբաղված է ճիշտ $n_i$ լսարան, $i=1,\ldots,r$: Իվենի, Իտայի և Շամիրի կողմից\footnote{S. Even, A. Itai, A. Shamir, On the complexity of timetable and multicommodity flow problems, SIAM J. Comput. 5 (4), 1976, pp. 691-703.} դիտարկվել է այնպիսի ուսումնական դասացուցակների կառուցումը, որտեղ հաշվի են առնվում ուսուցիչների նախապատվությունները: Ցույց է տրվել, որ ընդհանուր դեպքում խնդիրը NP-լրիվ է: Գրաֆների տեսության տերմիններով այս խնդիրը համապատասխանում է $G$ երկկողմանի մուլտիգրաֆի այնպիսի ճիշտ կողային ներկման կառուցմանը, որտեղ $G$-ի կողմերից մեկի յուրաքանչյուր գագաթի համար տրված են գույների բազմություններ, որոնցից պետք է ընտրվեն այդ գագաթին կից կողերի գույները: Դե Վերրայի կողմից\footnote{D. de Werra, Balanced schedules, INFOR. N9, 1971, pp. 230-237.} դիտարկվել են երկկողմանի մուլտիգրաֆների կողային ներկումների գոյության և կառուցման խնդիրները, երբ յուրաքանչյուր գագաթի համար այդ գագաթին կից ցանկացած երկու գույնով ներկված կողերի քանակների տարբերությունը մեծ չէ մեկից: Ցույց է տրվել, որ կամայական $G$ երկկողմանի մուլտիգրաֆի և $k$ բնական թվի համար $G$-ի կողերը կարելի է ներկել $k$ գույներով՝ բավարարելով նշված պայմանին: Այս խնդիրը համապատասխանում է այնպիսի դասացուցակներին, երբ ուսուցիչների և դասարանների ծանրաբեռնվածությունները բաշխվում են հավասարաչափ: «Պատուհան» չունեցող դասացուցակների գոյության և կառուցման խնդիրներին համապատասխանող գրաֆների ներկումների խնդիրների հետազոտման նպատակով Ա. Հասրաթյանի և Ռ. Քամալյանի կողմից\footnote{ А.С. Асратян, Р.Р. Камалян, Интервальные раскраски ребер мультиграфа, Прикладная математика, вып. 5, 1987, стр. 25-34.} սահմանվել է գրաֆի միջակայքային կողային ներկման գաղափարը: $G$ գրաֆի ճիշտ կողային ներկումը $1,\ldots,t$ գույներով կոչվում է $G$-ի միջակայքային կողային $t$-ներկում, եթե յուրաքանչյուր գագաթին կից կողերը ներկված են հաջորդական գույներով: Մի շարք աշխատանքներում հետազոտվել են երկկողմանի գրաֆների որոշ դասերի միջակայքային կողային ներկումները, որոնցից են լրիվ երկկողմանի գրաֆները, ծառերը, $n$-չափանի խորանարդը, ցանցերը, երկակի ուռուցիկ երկկողմանի գրաֆները, արտաքին հարթ երկկողմանի գրաֆները և $(a,b)$-երկհամասեռ երկկողմանի գրաֆները: Հայտնի է, որ ոչ բոլոր երկկողմանի գրաֆները ունեն միջակայքային կողային ներկում: Այդ փաստը առաջին անգամ նշվել է Ա. Միրումյանի կողմից: Հետագայում այդպիսի օրինակներ կառուցվել են Ս. Սեվաստյանովի, Պ. Էրդյոշի, Ա. Հերցի, Դ. դե Վերրայի, Կ. Գիարոյի, Մ. Կուբալի և Մ. Մալաֆիյսկու, ինչպես նաև Քամալյանի կողմից: Սեվաստյանովը ապացուցել է, որ երկկողմանի գրաֆների դասում միջակայքային կողային ներկման գոյության խնդիրը հանդիսանում է NP-լրիվ խնդիր\footnote{С.В. Севастьянов, Об интервальной раскрашиваемости ребер двудольного графа, Методы дискретного анализа в решении экстремальных задач, вып. 50, 1990, стр. 61-72.}: Նշենք նաև, որ Ջենսենի և Տոֆտի կողմից\footnote{T.R. Jensen, B. Toft, Graph coloring problems, Wiley Interscience Series in Discrete Mathematics and Optimization, 1995.} առաջարկվել է հիպոթեզ, համաձայն որի բոլոր $(a,b)$-երկհամասեռ երկկողմանի գրաֆները ունեն միջակայքային կողային ներկում: Այս հիպոթեզի ապացույցի ուղղությամբ Ա. Հասրաթյանը, Կ. Կասսելգրենը, Պ. Պետրոսյանը, Ջ. Վանդենբուշեն, Դ. Վեստը, Ա. Պյատկինը, Բ. Տոֆտը, Ֆ. Յանգը և Ք. Լին հասել են որոշ հաջողությունների:

Կիրառական խնդիրների մոդելավորման ժամանակ հաճախ հետազոտման օբյեկտներ են հանդիսանում ոչ միայն հասարակ գրաֆները, այլև մուլտիգրաֆները: Այսպես, օրինակ, ուսումնական դասացուցակների կառուցման խնդիրներում հաճախ առաջանում են իրավիճակներ, երբ ուսուցիչը միևնույն խմբի հետ անցկացնում է մեկից ավելի դասաժամ: Միջակայքային կողային ներկումներին նվիրված հետազոտությունները վերջին երեք տասնամյակներում հիմնականում վերաբերում էին պատիկ կողեր չպարունակող գրաֆներին, մինչդեռ մուլտիգրաֆները մնում են քիչ հետազոտված: Մյուս կողմից՝ վերոհիշյալ աշխատանքներում հիմնականում ուսումնասիրվել են միջակայքային ներկումների գոյության և կառուցման խնդիրները, սակայն քիչ է անդրադարձ կատարվել ներկումների թվային պարամետրերի գնահատման հարցերին: Միջակայքային ներկումների հետազոտություններում անարդարացիորեն քիչ է ուշադրություն հատկացված այդպիսի ներկումների կայունության հարցերին տարբեր գրաֆային գործողությունների նկատմամբ, ինչպիսիք են, օրինակ, գրաֆների գումարումը, դեկարտյան արտադրյալը, գրաֆից գագաթների, կողերի կամ զուգակցումների հեռացումը, կողերի տրոհումը և այլն: Նման դրվածքով խնդիրների կարևորությունը հատկապես մեծ է այնպիսի համակարգերի մոդելավորման դեպքում, որոնցում ժամանակի ազդեցության տակ կարող են կատարվել նկարագրող գրաֆի (մուլտիգրաֆի) չնախատեսված փոփոխություններ:

% Դիսկրետ մաթեմատիկայում մեծ ուշադրություն է հատկացվում ներկումների խնդիրների հետազոտություններին: Դա պայմանավորված է ինչպես ներկումների խնդիրների՝ մի շարք կարևոր կիրառական խնդիրների հետ առկա սերտ կապով, այնպես էլ նրանով, որ դիսկրետ մաթեմատիկայում առկա են բազմաթիվ խնդիրներ, որոնք կարելի է ձևակերպել որպես ներկումների խնդիրներ (ֆակտորիզացիայի խնդիրներ, տրոհման խնդիրներ, Ռամսեյի տեսության խնդիրներ և այլն): Մասնավորապես, նշանակալի փոխադարձ կապ կա կարգացուցակների տեսության խնդիրների և գրաֆների ներկումների խնդիրների միջև: Օրինակ, քննաշրջանի օպտիմալ կարգացուցակ կառուցելու խնդիրը բերվում է գրաֆի քրոմատիկ թվի որոշմանը: Գրաֆի քրոմատիկ դասը գտնելու խնդրին բերվում է սպորտային մրցումների կարգացուցակ կազմելու խնդիրը:

% «Պատուհան» չունեցող դասացուցակների գոյության և կառուցման խնդիրներին համապատասխանող գրաֆների ներկումների խնդիրների հետազոտման նպատակով Ա.Ս. Հասրաթյանի և Ռ.Ռ. Քամալյանի կողմից սահմանվել է գրաֆի միջակայքային կողային ներկման \cite{AsratianKamalian1987} գաղափարը: $G$ գրաֆի ճիշտ կողային ներկումը $1,2,\ldots,t$  գույներով կոչվում է $G$-ի միջակայքային կողային $t$-ներկում, եթե $\forall v \in V(G)$  գագաթին կից կողերը ներկված են հաջորդական գույներով և $\forall i$  գույնով, $1\leq i \leq t$ ներկված է $G$-ի գոնե մեկ կող: \cite{AsratianDenleyHaggvist1998,AsratianCasselgrenVandenbusscheWest2009,GiaroKubale1997,GiaroKubale2004,HansonLotenToft1998,Kamalian1989,Kamalian1990,KamalianMirumyan1997,KamalianPetrosyan2012,Kubale2004,PetrosyanKarapetyan2007,Petrosyan2010,PetrosyanKhachatrianTananyan2013,PetrosyanKhachatrian2014,Pyatkin2004,Sevastyanov1990,YangLi2011} աշխատանքներում հետազոտվել են երկկողմանի գրաֆների որոշ դասերի միջակայքային կողային ներկումները, որոնցից են լրիվ երկկողմանի գրաֆները, ծառերը, $n$-չափանի խորանարդը, ցանցերը, երկակի ուռուցիկ երկկողմանի գրաֆները,  երկհամասեռ երկկողմանի գրաֆները: Հայտնի է, որ ոչ բոլոր երկկողմանի գրաֆները ունեն միջակայքային կողային ներկում: Այդ փաստը առաջին անգամ նշվել է Ա.Ն. Միրումյանի կողմից: Հետագայում այդպիսի օրինակներ կառուցվել են Ս.Վ. Սեվաստյանովի \cite{Sevastyanov1990}, Պ. Էրդյոշի \cite{JensenToft1995}, Ա. Հերցի, Դ. դե Վերրայի \cite{JensenToft1995}, Պ. Պետրոսյանի և Հ. Խաչատրյանի \cite{PetrosyanKhachatrian2014} կողմից: \cite{Sevastyanov1990}-ում ապացուցվել է, որ երկկողմանի գրաֆների դասում միջակայքային կողային ներկման գոյության խնդիրը հանդիսանում է $NP$-լրիվ խնդիր: Նշենք նաև, որ \cite{JensenToft1995}-ում առաջարկվել է հիպոթեզ, համաձայն որի բոլոր  երկհամասեռ երկկողմանի գրաֆները ունեն միջակայքային կողային ներկում: Այս հիպոթեզի ապացույցի ուղղությամբ Ա.Ս. Հասրաթյանը, Կ. Կասսելգրենը, Ջ. Վանդենբուշեն, Դ. Վեստը, Ա. Պյատկինը, Ֆ. Յանգին և Ք. Լինը հասել են որոշ հաջողությունների \cite{AsratianCasselgrenVandenbusscheWest2009,Pyatkin2004,YangLi2011}: 

% Գրաֆների տարբեր արտադրյալներ ներմուծվել են Կ. Բերժի \cite{Berge1958}, Գ. Սաբիդուսսիի \cite{Sabidussi1960}, Ֆ. Հարարիի \cite{Harary1969} և Վ. Վիզինգի \cite{Vizing1963} կողմից: Մասնավորապես, Գ. Սաբիդուսսին \cite{Sabidussi1960} և Վ. Վիզինգը \cite{Vizing1963} ցույց են տվել, որ ցանկացած կապակցված գրաֆ միարժեքորեն ներկայացվում է նրա պարզ արտադրիչների դեկարտյան արտադրյալի միջոցով: Նշենք, որ գոյություն ունեն բավականին մեծ քանակությամբ հոդվածներ, որոնք նվիրված են գրաֆների տարբեր արտադրյալների կողային ներկումներին: Գրաֆների դեկարտյան արտադրյալների միջակայքային կողային ներկումները առաջին անգամ ուսումնասիրվել են Կ. Գիարոյի և Մ. Կուբալի կողմից \cite{GiaroKubale1997}: \cite{GiaroKubale2004}-ում Կ. Գիարոն և Մ. Կուբալը ապացուցել են, որ եթե $G$ և $H$ գրաֆները ունեն միջակայքային կողային ներկում, ապա $G \square H$ դեկարտյան արտադրյալը ևս ունի միջակայքային կողային ներկում: 2004 թվականին նույն հեղինակների կողմից \cite{GiaroKubale2004}-ում առաջարկվել է հիպոթեզ, համաձայն որի եթե $G$ և $H$ գրաֆները ունեն միջակայքային կողային ներկում, ապա $G[H]$ գրաֆների կոմպոզիցիան ևս ունի միջակայքային կողային ներկում: Այս հիպոթեզի ապացույցի ուղղությամբ Պ. Պետրոսյանը \cite{Petrosyan2011}-ում ապացուցել է, որ եթե $G$ և $H$ գրաֆները ունեն միջակայքային կողային ներկում և $H$-ը համասեռ գրաֆ է, ապա $G[H]$ գրաֆների կոմպոզիցիան ևս ունի այդպիսի ներկում: \cite{Yepremyan2011}-ում Լ. Եփրեմյանը ապացուցել է, որ այս հիպոթեզը ճիշտ է նաև այն դեպքում, երբ $G$-ն ծառ է, իսկ $H$-ը՝ պարզ շղթա կամ աստղ:

\paragraph{Աշխատանքի հիմնական նպատակը և նրանում դիտարկված խնդիրները:}
Աշխատանքում դիտարկվել են գրաֆների և մուլտիգրաֆների միջակայքային կողային ներկումների գոյության, կառուցման և թվային պարամետրերի գնահատման, ինչպես նաև այդպիսի ներկումների՝ գրաֆային տարբեր գործողությունների նկատմամբ կայունության խնդիրներ: Աշխատանքում նաև դիտարկվել են այդ ներկումների տարբեր բնույթի ընդհանրացումներ և ուսումնասիրվել են դրանց պարամետրերը: Աշխատանքի հիմնական նպատակն է վերոհիշյալ խնդիրների հետազոտումը գրաֆների և մուլտիգրաֆների տարբեր դասերի համար:


\paragraph{Հետազոտության օբյեկտները:}
Աշխատանքում հետազոտության օբյեկտներ են հանդիսանում գրաֆների և մուլտիգրաֆների տարբեր դասեր, գրաֆների հատուկ տիպի ֆակտորիզացիաներ, միջակայքային կողային ներկումներ, այդպիսի ներկումներում մասնակցող գույների քանակներ: Հետազոտության օբյեկտներ են հանդիսանում նաև միջակայքային ներկումներ չունեցող գրաֆներ և մուտլիգրաֆներ, ինչպես նաև այդպիսի գրաֆների՝ միջակայքային ներկվող գրաֆների դասից հեռավորության որոշ տեսակներ:


\paragraph{Հետազոտության մեթոդները:}
Հետազոտությունն իրականացվել է դիսկրետ մաթեմատիկայի, գրաֆների տեսության և դիսկրետ օպտիմիզացիայի մեթոդների օգնությամբ: Որոշ արդյունքների ստացման համար կիրառվել են համակարգչային հաշվարկների բաշխված համակարգեր:


\paragraph{Գիտական նորույթը:}
Աշխատանքում առաջին անգամ ուսումնասիրվել են մուլտիգրաֆների միջակայքային կողային ներկումների պարամետրեր, ներմուծվել է լրիվ գրաֆների հատուկ տիպի ֆակտորիզացիա (բաժանված) և նշվել է դրա կապը լրիվ գրաֆների միջակայքային կողային ներկումների հետ, տրվել է սեպարաբել միջակայքային կողային ներկման սահմանումը և կիրառվել է գրաֆների արտադրյալների վերաբերյալ մի շարք արդյունքների ստացման համար, ներմուծվել է միջակայքային ներկում չունեցող գրաֆների կառուցման ենթատրոհումների վրա հիմնված եղանակ, որի օգնությամբ կառուցվել է մինչ այժմ հայտնի այդպիսի ներկում չունեցող ամենափոքր գրաֆը:


\paragraph{Ստացված արդյունքների գործնական կիրառությունը:}
Աշխատանքում օգտագործված հետազոտության մեթոդները և նրանում ստացված արդյունքներն ունեն ոչ միայն տեսական կարևոր նշանակություն գրաֆների քրոմատիկ հատկությունների հետազոտման համար, այլև կարող են ունենալ գործնական կիրառություններ: Մասնավորապես, գրաֆների միջակայքային ներկումները կիրառվում են կոմպակտ կարգացուցակների գոյության և կառուցման, համակարգիչների և ծրագրերի հիշողության օպտիմալ բաշխման, անընդհատ պրոցեսների մաթեմատիկական մոդելավորման խնդիրներում և այլն: 



\paragraph{Պաշտպանության ներկայացվող հիմնական դրույթները:}
Պաշտպանության են ներկայացվում հետևյալ հիմնական դրույթները.

\begin{enumerate}
    \itemsep0em 
    \item Միջակայքային ներկումներ ունեցող գրաֆների և մուլտիգրաֆների պարամետրերի ընդհանուր գնահատականներ և որոշ դասերի գրաֆների համար այդ պարամետրերի ճշգրիտ արժեքներ,
    \item Լրիվ գրաֆների միջակայքային կողային ներկումների և այդ գրաֆների հատուկ տիպի ֆակտորիզացիաների համարժեքության հիման վրա ստացված արդյունքներ այդպիսի ներկումներում մասնակցող գույների հնարավոր քանակի վերաբերյալ,
    \item Լրիվ, լրիվ բազմակողմանի, արտաքին հարթ գրաֆների, ցանցերի, գլանների, տոռերի և Հեմինգի գրաֆների միջակայքային կողային ներկումների գոյության, կառուցման և թվային պարամետրերի գնահատականներ և ճշգրիտ արժեքներ,
    \item Կապակցված գրաֆների, համասեռ գրաֆների և երկկողմանի գրաֆների դեկարտյան արտադրյալների, ինչպես նաև հարթ դեկարտյան արտադրյալների միջակայքային կողային ներկումների գոյության, կառուցման, պարամետրերի գնահատման վերաբերյալ մի շարք արդյունքներ,
    \item Միջակայքային կողային ներկումների ընդհանրացումների հետ կապված որոշ պարամետրերի հասանելի գնահատականներ,
    \item Փոքրաթիվ գագաթներով երկկողմանի գրաֆների միջակայքային ներկումների կառուցման վերաբերյալ արդյունքներ, ինչպես նաև այդպիսի ներկում չունեցող փոքրագույն երկկողմանի գրաֆների և մուլտիգրաֆների կառուցման մի շարք եղանակներ, Ջենսեն-Տոֆտի միջակայքային ներկում չունեցող փոքրագույն երկկողմանի գրաֆի մասին պրոբլեմի մասնակի լուծում,
    \item Միջակայքային կողային ներկում չունեցող որոշ գրաֆների դեֆիցիտի ճշգրիտ արժեքներ, Բորովիցկա-Օլշեվսկայի, Դրգաշ-Բուրչարդտի և Հալուշչակի հիպոթեզի ապացույց:
\end{enumerate}

\paragraph{Ստացված արդյունքների գրաքննությունը և փորձարկումը:}

Ստացված արդյունքները զեկուցվել են մի շարք գիտաժողովներում Հայաստանում և եվրոպական երկրներում.
\begin{enumerate}
\itemsep0em 
\item Пятая годичная научная конференция РАУ, Ереван, Армения, 6-10 декабря 2010г.,
\item 14th Workshop on Graph Theory, Colourings, Independence and Domination, Szklarska Poreba, Poland, September 18-23, 2011,
\item 8th International Conference on Computer Science and Information Technologies, Yerevan, Armenia, September 26-30, 2011,
\item 15th Workshop on Graph Theory, Colourings, Independence and Domination, Szklarska Poreba, Poland, September 15-20, 2013,
\item 9th International Conference on Computer Science and Information Technologies, Yerevan, Armenia, September 23-27, 2013,
\item 7th Cracow Conference on Graph Theory, Rytro, Poland, September 14-19, 2014,
\item 5th Polish Combinatorial Conference, Bedlewo, Poland, September 22-26, 2014,
\item 8th Slovenian Conference on Graph Theory, Kranjska Gora, Slovenia, June 21-27, 2015,
\item 10th International Conference on Computer Science and Information Technologies, Yerevan, Armenia, September 28 - October 2, 2015:
\end{enumerate}

Աշխատանքի առանձին հատվածները մանրամասն քննարկվել են ավելի քան մեկ տասնյակ սեմինարների ընթացքում՝ Երևանի պետական համալսարանում, ՀՀ ԳԱԱ ինֆորմատիկայի և ավտոմատացման պրոբլեմների ինստիտուտում և Յագիլոնյան համալսարանում:

\paragraph{Հրապարակումները:}
Ատենախոսության թեմայի վերաբերյալ տպագրվել են 15 գիտական աշխատանքներ:


\paragraph{Աշխատանքի ծավալը և կառուցվածքը:}
Աշխատանքի ծավալը կազմում է 146 էջ: Աշխատանքը բաղկացած է ներածությունից, երեք գլուխներից, եզրակացությունից և գրականության ցանկից (83 անուն): Աշխատանքը ներառում է 31 նկար և 7 աղյուսակ:


