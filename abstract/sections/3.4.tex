Գիառոն\footnote{K. Giaro, Compact task scheduling on dedicated processors with no waiting periods, PhD thesis, Technical University of Gdansk, EIT faculty, Gdansk, 1999 (լեհերեն).} համակարգչային ծրագրի օգնությամբ ցույց է տվել, որ բոլոր երկկողմանի գրաֆները, որոնք ունեն առավելագույնը $14$ գագաթ միջակայքային ներկելի են: 3.4 պարագրաֆում նկարագրված են համակարգչային բաշխված համակարգերի միջոցով կատարված հաշվարկներ, որոնց միջոցով ցույց է տրվել, որ բոլոր $15$ և $16$ գագաթ ունեցող երկկողմանի գրաֆները միջակայքային ներկելի են: Ստացված արդյունքներից հետևում է, որ նվազագույն թվով գագաթներով միջակայքային ներկում չունեցող երկկողմանի գրաֆի գագաթների քանակը կարող է լինել $17$, $18$ կամ $19$:

\begin{hide}
\begin{lemma}
\label{mainLemma}
Եթե երկկողմանի գրաֆների որևէ $\mathfrak{F}$ բազմության համար $M(\mathfrak{F})$ և $C(\mathfrak{F})$ բազմությունների բոլոր գրաֆները միջակայքային ներկելի են, ապա $\mathfrak{F}$ բազմության բոլոր գրաֆները ևս միջակայքային ներկելի են:
\end{lemma}
\begin{proof}[Ապացույց]
Դիցուք $G$-ն $\mathfrak{F}$ բազմության որևէ երկկողմանի գրաֆ է: Եթե $G \in C(\mathfrak{F})$, ապա այն միջակայքային ներկելի է: Հակառակ դեպքում՝ այն կապակցված չէ, նվազագույն աստիճանը $1$ է, կամ կողմերից որևէ մեկում ունի $4$-ից քիչ գագաթներ: Եթե $G$-ն կապակցված չէ, ապա նրա կապակցված բաղադրիչներից յուրաքանչյուրը պատկանում է $M(\mathfrak{F})$-ին, ուստի $G$-ի ներկումը կստացվի կապակցված բաղադրիչների ներկումների միավորմամբ: Եթե գոյություն ունի $uv \in E(G)$ կախված կող, որտեղ $d_G(v)=1$, ապա կվերցնենք $G-v$ գրաֆի (որը պատկանում է $M(\mathfrak{F})$-ին) $\alpha$ ներկումը և $uv$ կողը կներկենք $\overline{S}(u,\alpha) + 1$ գույնով: Վերջապես, եթե $G$-ի կողմերից մեկում գագաթների թիվը փոքր է $4$-ից, ապա $G$-ն միջակայքային ներկելի է ըստ Թեորեմ \ref{t3_GiaroKubaleMalafiejski_min3}-ի:
\end{proof}
\begin{table}[t]
\renewcommand{\arraystretch}{1.2}
\begin{center}
\begin{tabular}{|c|c|c|}
\hline
Գագաթների քանակ & Գրաֆների քանակ & Պրոցեսորային ժամեր \\
\hline
4 / 11 & 16 308 & 3.04 \\
\hline
5 / 10 & 1 583 646 & 146.35 \\
\hline
6 / 9 & 43 739 172 & 340.51\\
\hline
7 / 8 & 243 304 742 & 15537.42\\
\hline
\end{tabular}
\end{center}
\caption{Թեորեմ \ref{th15}-ի ապացույցում սահմանված $C(\mathfrak{F})$ բազմության $15$ գագաթանի երկկողմանի գրաֆների ներկումների կառուցման համար կատարված հաշվարկների մանրամասներ: Պրոցեսորային ժամերը տրամադրվել են CrowdProcess-ի կողմից:}
\label{table15}
\end{table}

\begin{theorem}
\label{th15}
Բոլոր $15$ գագաթ ունեցող երկկողմանի գրաֆները միջակայքային ներկելի են:
\end{theorem}
\begin{proof}[Ապացույց]
Դիցուք $\mathfrak{F}$-ը $15$ գագաթ ունեցող բոլոր երկկողմանի գրաֆների բազմությունն է: $M(\mathfrak{F})$ բազմության բոլոր գրաֆները միջակայքային ներկելի են ըստ Թեորեմ \ref{t3_Giaro_14}-ի: Լեմմա \ref{mainLemma}-ի համաձայն, բավարար է ցույց տալ, որ $C(\mathfrak{F})$ բազմության բոլոր գրաֆները միջակայքային ներկելի են: $C(\mathfrak{F})$ բազմության գրաֆների թիվը 288 643 868 է: Այս գրաֆների ներկումները կառուցվել են վերը նկարագրված համակարգչային ալգորիթմի միջոցով: Կատարված հաշվարկների վերաբերյալ որոշ մանրամասներ ներկայացված են Աղյուսակ \ref{table15}-ում:
\end{proof}
\begin{table}[t]
\renewcommand{\arraystretch}{1.2}
\begin{center}
\begin{tabular}{|c|c|c|}
\hline
Գագաթների քանակ & Գրաֆների քանակ & Պրոցեսորային ժամեր \\
\hline
4 / 12 & 29 515 & 4.96 \\
\hline
4 / 13 & 51 616 & 19.19 \\
\hline
4 / 14 & 87 609 & 96.95\\
\hline
4 / 15 & 144 766 & N/A \\
\hline
\end{tabular}
\end{center}
\caption{$4$ գագաթ մի կողմում և $j$ ($12 \leq j \leq 15$) գագաթ մյուս կողմում պարունակող երկկողմանի գրաֆների ներկումների կառուցման համար կատարված հաշվարկների մանրամասներ: Պրոցեսորային ժամերը տրամադրվել են CrowdProcess-ի կողմից:}\label{table4x}
\end{table}

\begin{theorem}
\label{th4x}
Բոլոր այն երկկողմանի գրաֆները, որոնք ունեն $4$ գագաթ մի կողմում և ոչ ավել, քան $15$ գագաթ մյուս կողմում, միջակայքային ներկելի են, բացի նախորդ պարագրաֆում նկարագրված $\Delta_{5,5,5}$ գրաֆից (Նկ. \ref{f3_Shannon555}):
\end{theorem}
\begin{proof}[Ապացույց]
Դիցուք $\mathfrak{F}_{i,j}$-ն $(X,Y)$ կողմերով երկկողմանի գրաֆների բազմությունն է, որտեղ $|X|=i$ և $|Y| = j$, $i, j \in \mathbb{N}$: Նկատենք, որ $M(\mathfrak{F}_{i,j})=\bigcup\limits_{k=1}^{i-1}{\mathfrak{F}_{k,j}} \cup \bigcup\limits_{k=1}^{j-1}{\mathfrak{F}_{i,k}}$, ցանկացած $i,j>1$ թվերի համար: Պետք է ապացուցել, որ $\mathfrak{F}_{4,j}$, $12 \leq j \leq 15$, բազմությունների բոլոր գրաֆները միջակայքային ներկելի են, բացի Նկ. \ref{f3_Shannon555}-ում պատկերված $\Delta_{5,5,5}$ գրաֆից: Նկատենք, որ $\mathfrak{F}_{i,j}$, $i=1,2,3$, $j \in \mathbb{N}$, բազմությունների բոլոր գրաֆները միջակայքային ներկելի են համաձայն Թեորեմ \ref{t3_GiaroKubaleMalafiejski_min3}-ի: $\mathfrak{F}_{4,11}$ բազմության բոլոր գրաֆները միջակայքային ներկելի են ըստ Թեորեմ \ref{th15}-ի: $C(\mathfrak{F}_{4,j})$, $12 \leq j \leq 15$, բազմությունների բոլոր գրաֆները (բացառությամբ $\Delta_{5,5,5}$ գրաֆի) ներկելու համար կիրառում ենք վերը նկարագրված համակարգչային ծրագիրը: Հաշվարկների վերաբերյալ որոշ մանրամասներ ներկայացված են Աղյուսակ \ref{table4x}-ում: Ապացույցն ավարտելու համար հերթականորեն կիրառում ենք Լեմմա \ref{mainLemma}-ը $\mathfrak{F}_{4,j}$, $j=12,13,14,15$, բազմությունների համար:
\end{proof}

\begin{table}[t]
\renewcommand{\arraystretch}{1.2}
\begin{center}
\begin{tabular}{|c|c|c|}
\hline
Գագաթների քանակ & Գրաֆների քանակ \\
\hline
4 / 12 & 29 515  \\
\hline
5 / 11 & 5 158 975  \\
\hline
6 / 10 & 291 917 907 \\
\hline
7 / 9 & 3 604 370 591 \\
\hline
8 / 8 & 8 420 890 828 \\
\hline
\end{tabular}
\end{center}
\caption{Թեորեմ \ref{th16}-ի ապացույցում սահմանված $C(\mathfrak{F})$ բազմության $16$ գագաթանի երկկողմանի գրաֆների վիճակագրությունը: Հաշվարկները կատարվել են \cite{MamikonyanGithub} համակարգի միջոցով, որը չի տրամադրում պրոցեսորային ժամերի մասին տեղեկություններ:}
\label{table16}
\end{table}

\begin{theorem}
\label{th16}
Բոլոր $16$ գագաթ ունեցող երկկողմանի գրաֆները միջակայքային ներկելի են:
\end{theorem}
\begin{proof}[Ապացույց]
Ապացույցը կրկնում է Թեորեմ \ref{th15}-ի դատողությունները: Դիցուք $\mathfrak{F}$-ը $16$ գագաթ ունեցող բոլոր երկկողմանի գրաֆների բազմությունն է: $M(\mathfrak{F})$ բազմության բոլոր գրաֆները միջակայքային ներկելի են ըստ Թեորեմներ \ref{t3_Giaro_14}-ի և \ref{th15}-ի: Լեմմա \ref{mainLemma}-ի համաձայն, բավարար է ցույց տալ, որ $C(\mathfrak{F})$ բազմության բոլոր գրաֆները միջակայքային ներկելի են: $C(\mathfrak{F})$ բազմության գրաֆների թիվը 12 322 367 816 է: Այս գրաֆների ներկումները կառուցվել են վերը նկարագրված համակարգչային ալգորիթմի միջոցով: (Աղյուսակ \ref{table16})
\end{proof}
\end{hide}