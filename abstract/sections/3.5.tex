Աշխատանքի վերջին՝ 3.5 պարագրաֆում դիտարկվել են միջակայքային ներկում չունեցող երկկողմանի մուլտիգրաֆները: 
\begin{theorem}
\label{t3_bipartite_multi_4} Եթե $G$-ն կապակցված երկկողմանի մուլտիգրաֆ է, ընդ որում $\vert V(G)\vert\leq 4$, ապա $G\in \mathfrak{N}$: Մյուս կողմից՝ ցանկացած դրական $n \geq 5$ թվի համար գոյություն ունի $G$ կապակցված երկկողմանի մուլտիգրաֆ, որի համար $|V(G)|=n$ և $G\notin \mathfrak{N}$:
\end{theorem}
\begin{proof}[Ապացույց]
Այն դեպքերը, երբ $\vert V(G)\vert\leq 3$ ակնհայտ են: Ենթադրենք $\vert
V(G)\vert= 4$. Եթե $G$-ի հիմքում ընկած գրաֆը ծառ է, ապացույցը նորից ակնհայտ է: Ուստի, ենթադրենք $V(G)=\{u,v,w,z\}$ և
$E(G)=E(uv)\cup E(vw)\cup E(wz)\cup E(uz)$, որտեղ $\mu(uv)=a$,
$\mu(vw)=b$, $\mu(wz)=c$, $\mu(uz)=d$: Առանց ընդհանրությունը խախտելու կարող ենք ենթադրել, որ ${\max} \{a,b,c,d\}=d$. Ներկենք $E(uv)$ կողերը
$d+1,\ldots,d+a$ գույներով, $E(vw)$ կողերը՝ 
$d-b+1,\ldots,d$, $E(wz)$ կողերը՝ $d+1,\ldots,d+c$,
իսկ $E(uz)$ կողերը՝ $1,\ldots,d$ գույներով: Եթե $a<c$, ապա ստացվածը $G$ մուլտիգրաֆի միջակայքային $(d+c)$-ներկում է: Հակառակ դեպքում, ստացվածը $G$ մուլտիգրաֆի միջակայքային $(d+a)$-ներկում է:

Թեորեմի երկրորդ մասը ցույց տալու համար նկատենք, որ ըստ Թեորեմ \ref{t3_parachute}-ի` ցանկացած $n \geq 3$ թվի համար $G_n = Par(\underbrace{n+1,\ldots,n+1}_{n})$ երկկողմանի մուլտիգրաֆը միջակայքային ներկում չունի, ընդ որում՝ $|V(G_n)|=n+2$:
\end{proof}

\begin{theorem}
\label{t3_bipartite_multi_Delta3} Եթե $G$-ն երկկողմանի մուլտիգրաֆ է, որի համար
$\Delta(G)\leq 3$, ապա $G\in \mathfrak{N}$ և $w(G)\leq 4$: Մյուս կողմից՝ ցանկացած դրական $\Delta \geq 9$ թվի համար գոյություն ունի $G$ կապակցված երկկողմանի մուլտիգրաֆ, որի համար $\Delta(G)=\Delta$ և $G\notin \mathfrak{N}$:
\end{theorem}
\begin{proof}[Ապացույց] Ցույց տանք, որ եթե $G$-ն երկկողմանի մուլտիգրաֆ է, ընդ որում $\Delta(G)\leq 3$, ապա
$G$-ն ունի միջակայքային ներկում ոչ ավել քան չորս գույներով:

Ապացույցը կատարենք ինդուկցիայով ըստ $\vert E(G)\vert$-ի: Պնդումն ակնհայտ է, երբ $\vert E(G)\vert\leq 4$:
Ենթադրենք, որ $\vert E(G)\vert\geq 5$ և պնդումը ճիշտ է բոլոր այնպիսի $G^{\prime}$ երկկողմանի մուլտիգրաֆների համար, 
որոնց համար $\Delta(G^{\prime})\leq 3$ և $\vert
E(G^{\prime})\vert<\vert E(G)\vert$:

Դիտարկենք $G$ մուլտիգրաֆը: Այն կապակցված է: Եթե
$\Delta(G)\leq 2$, ապա $G\in \mathfrak{N}$ և $w(G)\leq 2$. Ենթադրենք, որ $\Delta(G)=3$. Եթե $G$-ն չունի պատիկ կողեր, ապա պնդումը հետևում է Թեորեմ \ref{t3_Hansen_Delta3}-ից: Ուստի ենթադրենք, որ $G$-ն ունի պատիկ կողեր:

Դիցուք $uv\in E(G)$ և $\mu(uv)\geq 2$. Եթե $\mu(uv)= 3$, ապա $G$-ն բաղկացած է միայն $u$ և $v$ գագաթներից և $uv$ երեք պատիկ կողերից, ուստի այն ունի միջակայքային $3$-ներկում: Այժմ ենթադրենք, որ $\mu(uv)= 2$: Դիտարկենք երկու դեպք.

Դեպք 1. $d_{G}(v)=2$ և $d_{G}(u)=\Delta(G)=3$:

Այս դեպքում գոյություն ունի $uw$ կող, որը հանդիսանում է կամուրջ
$G$-ում: Դիտարկենք $G^{\prime}=G-E(uv)$ մուլտիգրաֆը, որտեղ
$E(uv)=\{e_{1},e_{2}\}$. Ինդուկցիոն ենթադրության համաձայն, $G^{\prime}$-ը ունի $\alpha$ միջակայքային ներկում ոչ ավել քան չորս գույներով:
Առանց ընդհանրությունը խախտելու կարող ենք համարել, որ $\alpha (uw)\leq 2$ (հակառակ դեպքում կդիտարկենք $\beta(e)=4-\alpha(e)$ կամ
$\beta(e)=5-\alpha(e)$ ներկումը ցանկացած $e\in E(G^{\prime})$ կողի համար): Այժմ ներկենք $e_{i}$ կողը $\alpha (uw)+i$, գույնով, $i=1,2$: Դժվար չէ համոզվել, որ ստացված ներկումը $G$ մուլտիգրաֆի միջակայքային ներկում է ոչ ավել քան չորս գույներով:

Դեպք 2. $d_{G}(u)=d_{G}(v)=\Delta(G)=3$:

Այս դեպքում $G$-ում գոյություն ունեն $x,y$ գագաթներ ($x\neq y$), այնպես, որ $ux\in E(G)$ և $vy\in E(G)$: Դիտարկենք $G^{\prime}=(G-E(uv)-ux-vy)+xy$ մուլտիգրաֆը, որտեղ $E(uv)=\{e_{1},e_{2}\}$: Ինդուկցիոն ենթադրության համաձայն, $G^{\prime}$-ը ունի $\alpha$ միջակայքային ներկում ոչ ավել քան չորս գույներով:
Առանց ընդհանրությունը խախտելու կարող ենք համարել, որ $\alpha (xy)\leq 2$ (հակառակ դեպքում կդիտարկենք $\beta(e)=4-\alpha(e)$ կամ
$\beta(e)=5-\alpha(e)$ ներկումը ցանկացած $e\in E(G^{\prime})$ կողի համար): Այնուհետև ջնջենք $xy$ կողը և ներկենք $ux$ և $vy$ կողերը
$\alpha (xy)$ գույնով, իսկ $e_{i}$ կողը՝ $\alpha (xy)+i$ գույնով,
$i=1,2$: Հեշտ է տեսնել, որ ստացված ներկումը $G$ մուլտիգրաֆի միջակայքային ներկում է ոչ ավել քան չորս գույներով:

Թեորեմի երկրորդ մասի ապացույցը անմիջապես հետևում է Պնդում \ref{r3_parachute_333}-ից և Թեորեմ \ref{t3_parachute}-ից:
\end{proof}