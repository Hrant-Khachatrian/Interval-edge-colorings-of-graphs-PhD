
Աշխատանքում դիտարկվում են ոչ կողմնորոշված հասարակ գրաֆներ՝ առանց պատիկ կողերի և օղերի, ինչպես նաև մուլտիգրաֆներ, որտեղ թույլատրվում են պատիկ կողեր: $G$ գրաֆի (մուլտիգրաֆի) գագաթների և կողերի բազմությունները նշանակենք, համապատասխանաբար, $V(G)$-ով և $E(G)$-ով: Ցանկացած $v\in V(G)$-ի համար $d_G(v)$-ով նշանակենք այդ գագաթի աստիճանը $G$-ում, $\delta(G)$-ով և $\Delta(G)$-ով նշանակենք գրաֆի (մուլտիգրաֆի) նվազագույն և առավելագույն աստիճանները:

Ցանկացած $u,v\in V(G)$ գագաթների համար $d(u,v)$-ով նշանակենք $u$ և $v$ գագաթների միջև հեռավորությունը $G$ գրաֆում (մուլտիգրաֆում): $v \in V(G)$ գագաթի համար սահմանենք $\epsilon(v)$ թիվը հետևյալ կերպ.
$\epsilon(v) = \max_{u\in V(G)}{d(u,v)}$,
իսկ $G$ գրաֆի (մուլտիգրաֆի) տրամագիծը՝ 
$\mathrm{diam}(G) = \max_{v\in V(G)}{\epsilon(v)}$:

$\alpha'(G)$-ով կնշանակենք $G$ մուլտիգրաֆի ամենաշատ կողեր պարունակող զուգակցման հզորությունը: $\mathfrak{F} = \left\{F_1,\ldots,F_n \right\}$ կատարյալ զուգակցումների բազմությունը կանվանենք $G$ մուլտիգրաֆի $1$-ֆակտորիզացիա, եթե $G$-ի կամայական կող պատկանում է $\mathfrak{F}$-ի զուգակցումներից ճիշտ մեկին:

$\alpha : E(G) \rightarrow \mathbb{N}$ ֆունկցիան կոչվում է $G$ մուլտիգրաֆի ճիշտ կողային ներկում, եթե $\forall v \in V(G)$ գագաթին կից կողերը ներկված են զույգ առ զույգ տարբեր գույներով:
Եթե $\alpha$ ճիշտ կողային ներկումը օգտագործում է միայն $1,\ldots,t$ գույները, ընդ որում՝ $\forall i (1 \leq i \leq t)$ համար $\exists e_i \in E(G)$ այնպիսին, որ $\alpha(e_i)=i$, $\alpha$-ն կանվանենք $G$ մուլտիգրաֆի ճիշտ կողային $t$-ներկում: Եթե $\alpha$-ն $G$-ի ճիշտ կողային ներկում է, ապա կամայական $v$ գագաթի սպեկտրը՝ $S(v,\alpha)$, այդ գագաթին կից կողերի գույների բազմությունն է: Սպեկտրի նվազագույն և առավելագույն թվերը կնշանակենք հետևյալ կերպ. $\underline{S}(v, \alpha) = \min{S(v,\alpha)}$, $\overline{S}(v, \alpha) = \max{S(v,\alpha)}$:

Տրված $G$ մուլտիգրաֆի ճիշտ կողային ներկումներում անհրաժեշտ գույների նվազագույն քանակը կոչվում է քրոմատիկ դաս և նշանակվում է $\chi'(G)$-ով: Ըստ Վիզինգի հայտնի թեորեմի, $\Delta(G) \leq \chi'(G) \leq \Delta(G) + \mu(G)$, որտեղ $\mu(G)$-ն $G$ գրաֆում կողերի առավելագույն պատիկությունն է: 

$\alpha$ ճիշտ կողային $t$-ներկումը կանվանենք միջակայքային կողային $t$-ներկում, եթե ցանկացած $v \in V(G)$ գագաթի համար $S(v,\alpha)$ բազմությունը միջակայք է: Նշանակենք $\mathfrak{N}_t$-ով այն գրաֆների բազմությունը, որոնք ունեն միջակայքային կողային $t$-ներկում, իսկ $\mathfrak{N}$-ով՝ $\mathfrak{N}=\bigcup_{t\geq1}{\mathfrak{N}_t}$ բոլոր միջակայքային կողային ներկելի գրաֆների բազմությունը: Երբ $G \in \mathfrak{N}$, $G$-ի միջակայքային կողային ներկման մեջ օգտագործվող գույների նվազագույն և առավելագույն քանակները նշանակենք, համապատասխանաբար, $w(G)$-ով և $W(G)$-ով\footnote{Չսահմանված հասկացությունների և նշանակումների համար տե՛ս D.B. West, Introduction to Graph Theory, Prentice-Hall, New Jersey, 1996.}: