Աշխատանքում դիտարկվել են մուլտիգրաֆների տարբեր դասերի՝ $\mathfrak{N}$ դասին պատկանելու խնդիրներ, այդ դասին պատկանող մուլտիգրաֆների համար՝ $w(G)$ և $W(G)$ պարամետրերի գնահատման և ճշգրիտ արժեքների որոշման խնդիրներ, միջակայքային ներկումներում մասնակցող գույների հնարավոր քանակի որոշման խնդիրներ, գրաֆների տարբեր գործողությունների նկատմամբ միջակայքային ներկելիության կայունության խնդիրներ, ինչպես նաև ուսումնասիրվել են միջակայքային ներկում չունեցող գրաֆների դեֆիցիտը և $w_{def}(G)$ ու $W_{def}(G)$ պարամետրերը:

Աշխատանքում ստացվել են հետևյալ արդյունքները:
\begin{enumerate}
\itemsep0em 
\item Միջակայքային ներկումներ ունեցող գրաֆների և մուլտիգրաֆների $w(G)$ և $W(G)$ պարամետրերի, ինչպես նաև միջակայքային ներկում չունեցող գրաֆների $w_{def}(G)$ և $W_{def}(G)$ պարամետրերի համար տրվել են հասանելի գնահատականներ,

\item $K_{2n}$ լրիվ գրաֆների միջակայքային կողային ներկումների և այդ գրաֆների հատուկ տիպի ֆակտորիզացիաների համարժեքության հիման վրա ստացվել են  $W(K_{2n})$ պարամետրի նոր ստորին և վերին գնահատականներ, գտնվել են այդ պարամետրի ճշգրիտ արժեքները $n \leq 12$ արժեքների համար,

\item Ստացվել են լրիվ բազմակողմանի գրաֆների, արտաքին հարթ գրաֆների, ցանցերի, գլանների, տոռերի և Հեմինգի գրաֆների միջակայքային կողային ներկումների գոյության վերաբերյալ արդյունքներ, $w(G)$ և $W(G)$ պարամետրերի գնահատականներ, որոշ դեպքերում՝ նաև ճշգրիտ արժեքներ,

\item Ցույց է տրվել գրաֆների միջակայքային ներկելիության կապը այդ գրաֆների մասնակցությամբ դեկարտյան արտադրյալների միջակայքային ներկելիություն հետ, էապես ուժեղացվել են համասեռ գրաֆների մասնակցությամբ դեկարտյան արտադրյալների համար $W(G \square H)$ պարամետրի հայտնի գնահատականները, ապացուցվել է երկկողմանի գրաֆների մասնակցությամբ դեկարտյան արտադրյալների մի շարք դասերի միջակայքային ներկելիությունը,

\item Գտնվել են որոշ գրաֆների դեֆիցիտի ճշգրիտ արժեքները, հաստատվել է Բորովիցկա-Օլշեվսկայի, Դրգաշ-Բուրչարդտի և Հալուշչակի հիպոթեզը, ցույց է տրվել, որ արտաքին հարթ գրաֆները բավարարում են դեֆիցիտի մասին հիպոթեզին, համաձայն որի ցանկացած $G$ գրաֆի համար $def(G) \leq |V(G)|$,

\item Մասնակի լուծում է տրվել միջակայքային ներկում չունեցող փոքրագույն երկկողմանի գրաֆների մասին Ջենսեն-Տոֆտի խնդրին, կառուցվել են այդպիսի գրաֆների և մուլտիգրաֆների հայտնի փոքրագույն օրինակները, համակարգչային հաշվարկների միջոցով ցույց է տրվել, որ ոչ ավել, քան 16 գագաթ պարունակող բոլոր երկկողմանի գրաֆները ունեն միջակայքային ներկումներ:
\end{enumerate}