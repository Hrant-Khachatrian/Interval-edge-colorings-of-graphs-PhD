\begin{frame}{Երկկողմանի գրաֆների դեկարտյան արտադրյալներ}
\begin{itemize}
\item Երկկողմանի գրաֆների միջակայքային ներկելիությունը պարզելը NP-լրիվ խնդիր է (Սեվաստյանով, 1990): 

% \item Մինչ այժմ բաց է մնում 4 առավելագույն աստիճան ունեցող երկկողմանի գրաֆների միջակայքային ներկելիության հարցը (Ջենսեն, Տոֆտ, 1995):
\end{itemize}

\begin{theorem}[2.3.3]
Եթե $G$-ն երկկողմանի գրաֆ է, որի համար $\Delta(G)\leq
4$, ապա $G\square K_{2}\in \mathfrak{N}$:
\end{theorem}
\end{frame}

\begin{frame}{Երկկողմանի գրաֆների դեկարտյան արտադրյալներ}

\begin{theorem}[2.3.5]
Եթե $G$-ն երկկողմանի գրաֆ է, որի համար $\Delta(G) =
5$, և չունի $3$ աստիճան ունեցող գագաթ, ապա $G\square K_{2}\in \mathfrak{N}$:
\end{theorem}

\begin{theorem}[2.3.7]
Եթե $G$-ն կատարյալ զուգակցում ունեցող երկկողմանի գրաֆ է, որի համար $\Delta(G)=5$, ապա $G \square K_2 \in \mathfrak{N}$:
\end{theorem}


\begin{theorem}[2.3.9]
Եթե $G$-ն երկկողմանի գրաֆ է, որի համար $\Delta(G)=6$ և ունի 2-ֆակտոր, ապա $G \square K_2 \in \mathfrak{N}$:
\end{theorem}
\end{frame}