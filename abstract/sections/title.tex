\thispagestyle{empty}
\begin{center}

\textsc{ԵՐԵՎԱՆԻ ՊԵՏԱԿԱՆ ՀԱՄԱԼՍԱՐԱՆ}\\[1cm]

{Խաչատրյան Հրանտ Հարությունի\\[0.5cm]}
{Գրաֆների միջակայքային կողային ներկումների հետազոտում\\[1.2cm]}
{ՍԵՂՄԱԳԻՐ \\[1.2cm]}

Ա.01.09 «Մաթեմատիկական կիբեռնետիկա և  
մաթեմատիկական \\ տրամաբանություն» մասնագիտությամբ ֆիզիկամաթեմատիկական \\ գիտությունների թեկնածուի գիտական աստիճանի հայցման ատենախոսության \\[1cm]

Երևան - 2017\\[0.6cm]
\rule{\textwidth}{1pt}\\[0.8cm]

\textsc{ЕРЕВАНСКИЙ ГОСУДАРСТВЕННЫЙ УНИВЕРСИТЕТ}\\[1cm]

{Xaчатрян Грант Арутюнович\\[0.5cm]}
{Исследование интервальных реберных раскрасок графов\\[1.2cm]}
{АВТОРЕФЕРАТ \\[1.2cm]}

диссертации на соискание ученой степени кандидата \\
физико-математических наук по специальности\\ 01.01.09 “Математическая кибернетика и математическая логика” \\[1cm]
\vfill

Ереван - 2017\\[0.2cm]


\end{center}

Ատենախոսության թեման հաստատվել է Երևանի պետական համալսարանում:\\[0.3cm]

\begin{tabularx}{\textwidth}{Xl}
Գիտական ղեկավար՝     & ֆիզ.-մաթ. գիտ. թեկնածու Պ. Ա. Պետրոսյան \\
Պաշտոնական ընդդիմախոսներ՝ & ֆիզ.-մաթ. գիտ. դոկտոր Է. Մ. Պողոսյան \\
 & ֆիզ.-մաթ. գիտ. թեկնածու Հ. Ց. Հակոբյան \\
Առաջատար կազմակերպություն՝  & Հայ-ռուսական (սլավոնական) համալսարան
\end{tabularx}\\[0.5cm]
     
Պաշտպանությունը կայանալու է 2017թ. հունիսի 9-ին, ժ. 14\textsuperscript{30}-ին ԵՊՀ-ում գործող ԲՈՀ-ի 044 «Մաթեմատիկական կիբեռնետիկա» մասնագիտական խորհրդի նիստում հետևյալ հասցեով՝ 0025, Երևան, Ալ. Մանուկյան 1:\\[0.5cm]  
Ատենախոսությանը կարելի է ծանոթանալ ԵՊՀ-ի գրադարանում: \\[0.5cm]
Սեղմագիրն առաքված է 2017թ. մայիսի 6-ին:\\[0.5cm]
\begin{tabularx}{\textwidth}{Xlr}
Մասնագիտական խորհրդի  & \multirow{3}{*}{\includegraphics[width=0.1\textwidth]{figures/dumanyan.pdf}\hspace{1cm}} &                \\
գիտական քարտուղար,    &                                                                                    &                \\
ֆիզ.-մաթ.գիտ. դոկտոր՝ &                                                                                    & Վ.Ժ. Դումանյան
\end{tabularx}\\[0.5cm]

\rule{\textwidth}{1pt}\\[0.4cm]

Тема диссертации утверждена в Ереванском государственном университете.\\[0.3cm]

\begin{tabularx}{\textwidth}{Xl}
Научный руководитель:     & кандидат физ.-мат. наук П. А. Петросян \\
Официальные оппоненты:  & доктор физ.-мат. наук Э. М. Погосян \\
 & кандидат физ.-мат. наук Г. Ц. Акопян \\
Ведущая организация:  & Российско-Армянский (Славянский) университет
\end{tabularx}\\[0.5cm]
                          
Защита состоится 9-го июня 2017г. в 14\textsuperscript{30} часов на заседании действующего в Ереванском государственном университете специализированного совета ВАК 044
“Математическая кибернетика”, по адресу: Ереван 0025, ул. А. Манукяна, 1.\\[0.5cm]
С диссертацией можно ознакомиться в библиотеке Ереванского государственного университета.\\[0.5cm]
Автореферат разослан 6-го мая 2017г. \\[0.5cm]

\begin{tabularx}{\textwidth}{Xlr}
Ученый секретарь  & \multirow{3}{*}{\includegraphics[width=0.1\textwidth]{figures/dumanyan.pdf}\hspace{1cm}} &                \\
специализированного  совета,    &                                                                                    &                \\
доктор физ.-мат. наук &                                                                                    & В.Ж. Думанян
\end{tabularx}


\pagebreak