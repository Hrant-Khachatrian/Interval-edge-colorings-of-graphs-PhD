Հայտնի է, որ երկկողմանի գրաֆների միջակայքային ներկելիությունը պարզելը NP-լրիվ խնդիր է: Մինչ այժմ բաց է մնում 4 առավելագույն աստիճան ունեցող երկկողմանի գրաֆների միջակայքային ներկելիության հարցը: 2.3 պարագրաֆում դիտարկվել են $K_2$ գրաֆի և փոքր առավելագույն աստիճանով երկկողմանի գրաֆների դեկարտյան արտադրյալներ: Ցույց է տրվել, որ եթե $G$-ն երկկողմանի գրաֆ է, ապա $G \square K_2 \in \mathfrak{N}$, երբ $\Delta(G)=4$, $\Delta(G)=5$ և $G$-ն չունի երեք աստիճան ունեցող գագաթ, $\Delta(G)=5$ և $G$-ն ունի կատարյալ զուգակցում, կամ $\Delta(G)=6$ և $G$-ն ունի 2-ֆակտոր:

\begin{hide}
\begin{theorem}
\label{t2_cartesian_gap} $G\in \mathfrak{N}^{h}$ այն և միայն այն դեպքում, երբ $G\square
Q_{h}\in \mathfrak{N}$:
\end{theorem}
\begin{proof}[Ապացույց]
Դիցուք $\alpha$-ն $G$-ի միջակայքային $(t,h)$-ներկում է: Կառուցենք $G\square Q_{h}$ գրաֆի $\beta$ ներկումը հետևյալ կերպ: Նախ $G\square Q_{h}$-ի յուրաքանչյուր $G$-շերտ ներկենք $\alpha$ ներկման համաձայն: Այնուհետև, կամայական $u\in V(G)$ գագաթի համար, ներկենք $G\square Q_{h}$-ի համապատասխան $Q_{h}$-շերտը օգտագործելով $\overline
{S\left(u,\alpha \right)}$ բազմությունից $h$ գույներ: Պարզ է, որ ցանկացած $(u,v)\in V(G\square
Q_{h})$ գագաթի համար, $S\left((u,v),\beta \right)=S\left(u,\alpha \right)\cup
\overline{S\left(u,\alpha \right)}=\left[\underline S\left(u,\alpha
\right),\underline S\left(u,\alpha \right)+d_{G}(u)+h-1\right]$:
Այստեղից հետևում է $G\square Q_{h}\in \mathfrak{N}$:

Դիցուք $\gamma$-ն $G\square
Q_{h}$-ի միջակայքային $t^{\prime}$-ներկում է: Այս կողային ներկման սահմանափակումը $G\square Q_{h}$-ի որևէ $G$-շերտի կողերի վրա կարող է ձևափոխվել $G$-ի
միջակայքային $(t^{\prime\prime},h)$-ներկման, որտեղ $t^{\prime\prime}\leq
t^{\prime}$:
\end{proof} % բացել ապացույցը

\begin{theorem}
\label{t2_bipartite_Delta4} Եթե $G$-ն երկկողմանի գրաֆ է, որի համար $\Delta(G)=4$,
ապա $G\in \mathfrak{N}^{1}$ և $w^{1}(G)=4$:
\end{theorem}
\begin{proof}[Ապացույց] Դիցուք $G$-ն երկկողմանի գրաֆ է 4 առավելագույն աստիճանով:
Հայտնի է, որ եթե $G$-ն չունի 3 աստիճանով գագաթ, ապա այն ունի միջակայքային $4$-ներկում \cite{Giaro1997}: Ուստի, $G\in
\mathfrak{N}^{1}$ և $w^{1}(G)=4$: Այժմ ենթադրենք, որ $G$-ն ունի 3 աստիճանով որոշ գագաթներ: Կառուցենք նոր $G^{\star}$ գրաֆը հետևյալ կերպ. $G$-ի յուրաքանչյուր $3$ աստիճան ունեցող գագաթին ավելացնենք կախված կող: Հեշտ է տեսնել, որ $G^{\star}$-ը երկկողմանի գրաֆ է 4 առավելագույն աստիճանով և չունի 3 աստիճանով գագաթներ, ուստի, այն ևս ունի միջակայքային $4$-ներկում: Այժմ դիտարկենք այս միջակայքային $4$-ներկման սահմանափակումը $G$-ի կողերի վրա: Պարզ է, որ այս ներկումը $G$-ի միջակայքային $(4,1)$-ներկում է:
\end{proof}

\begin{corollary}
\label{c2_bipartite_Delta4} Եթե $G$-ն երկկողմանի գրաֆ է, որի համար $\Delta(G)\leq
4$, ապա $G\square K_{2}\in \mathfrak{N}$:
\end{corollary}

\begin{theorem}
\label{t2_bipartite_Delta5_no3} Եթե $G$-ն երկկողմանի գրաֆ է, որի համար $\Delta(G)=5$ և չունի $3$ աստիճան ունեցող գագաթ, ապա $G\in \mathfrak{N}^{1}$ և
$w^{1}(G)=5$:
\end{theorem}
\begin{proof}[Ապացույց]
Դիցուք $G$-ն երկկողմանի գրաֆ է, որի առավելագույն աստիճանը $5$ է,
բայց չունի $3$ աստիճան ունեցող գագաթ: Ըստ Հոլլի թեորեմի, $G$-ն ունի առավելագույն աստիճան ունեցող գագաթները ծածկող զուգակցում: Դիցուք $M$-ը $G$-ի այդպիսի զուգակցում է: Դիտարկենք $G^{\prime}=G-M$ գրաֆը:
Պարզ է, որ $G^{\prime}$-ը երկկողմանի գրաֆ է, որի համար
$\Delta(G^{\prime})=4$: Ինչպես Թեորեմ \ref{t2_bipartite_Delta4}-ի ապացույցում, կարող ենք ցույց տալ, որ $G^{\prime}$-ը ունի $\alpha$ միջակայքային $(4,1)$-ներկում այնպիսին, որ կամայական $v\in
V(G^{\prime})$ գագաթի համար, որի համար $d_{G^{\prime}}(v)\in \{1,2,4\}$, $S(v,\alpha)$-ն ամբողջ թվերի միջակայք է: Այժմ կառուցենք $G^{\prime}$-ի նոր $\beta$ ներկումը վերցնելով $\alpha$ ներկման գույները և փոխարինելով 3 և 4 գույները համապատասխանաբար 4-ով և 5-ով: Այսպիսով, ամեն $v\in
V(G^{\prime})$ գագաթի համար կունենանք, որ

$\bullet$ եթե $d_{G^{\prime}}(v)=4$, ապա $S(v,\beta)=
\{1,2,4,5\}$,

$\bullet$ եթե $d_{G^{\prime}}(v)=3$, ապա
$S(v,\beta)\in \{\{1,2,4\},\{1,2,5\},\{1,4,5\},\{2,4,5\}\}$,

$\bullet$ եթե $d_{G^{\prime}}(v)=2$, ապա $S(v,\beta)\in
\{\{1,2\},\{2,4\},\{4,5\}\}$,

$\bullet$ եթե $d_{G^{\prime}}(v)=1$, ապա $S(v,\beta)\in
\{\{1\},\{2\},\{4\},\{5\}\}$:

Այժմ սահմանենք $G$-ի $\gamma$ կողային ներկումը հետևյալ կերպ.

1) կամայական $e\in E(G^{\prime})$ կողի համար $\gamma(e)=\beta(e)$,

2) կամայական $e\in M$ կողի համար $\gamma(e)=3$:

Քանի որ $G$-ն չունի $3$ աստիճան ունեցող գագաթ, ապա կամայական $v\in V(G)$ գագաթի համար ստանում ենք, որ

$\bullet$ եթե $d_{G}(v)=5$, ապա $S(v,\gamma)=[1,5]$,

$\bullet$ եթե $d_{G}(v)=4$, ապա $S(v,\gamma)\in
\{[1,4],[2,5],\{1,2,3,5\},\{1,2,4,5\},\{1,3,4,5\}\}$,

$\bullet$ եթե $d_{G}(v)=2$, ապա
$S(v,\gamma)\in
\{\{1,2\},\{2,4\},\{4,5\},\{1,3\},\{2,3\},\{3,4\},\{3,5\}\}$,

$\bullet$ եթե $d_{G}(v)=1$, ապա $S(v,\gamma)\in
\{\{1\},\{2\},\{3\},\{4\},\{5\}\}$:

Այստեղից հետևում է, որ $\gamma$-ն $G$-ի միջակայքային $(5,1)$-ներկում է:
\end{proof}

\begin{corollary}
\label{c2_bipartite_Delta5_no3} Եթե $G$-ն երկկողմանի գրաֆ է, որի համար $\Delta(G) =
5$, և չունի $3$ աստիճան ունեցող գագաթ, ապա $G\square K_{2}\in \mathfrak{N}$:
\end{corollary}

\begin{theorem}
\label{t2_bipartite_Delta5_nopm} Եթե $G$-ն կատարյալ զուգակցում ունեցող երկկողմանի գրաֆ է, որի համար $\Delta(G)=5$, ապա $G\in \mathfrak{N}^{1}$ և
$w^{1}(G)=5$:
\end{theorem}
\begin{proof}[Ապացույց]
Դիտարկենք $G' = G - M$ գրաֆը, որտեղ $M$-ը $G$-ում առկա որևէ կատարյալ զուգակցում է: Թեորեմ \ref{t2_bipartite_Delta5_no3}-ի ապացույցում նկարագրված եղանակով կարելի է կառուցել $G'$-ի $\beta$ ճիշտ կողային ներկում $1,2,4,5$ գույներով այնպես, որ ցանկացած $v\in
V(G^{\prime})$ գագաթի համար կունենանք հետևյալը.

$\bullet$ եթե $d_{G^{\prime}}(v)=4$, ապա $S(v,\beta)=
\{1,2,4,5\}$,

$\bullet$ եթե $d_{G^{\prime}}(v)=3$, ապա
$S(v,\beta)\in \{\{1,2,4\},\{1,2,5\},\{1,4,5\},\{2,4,5\}\}$,

$\bullet$ եթե $d_{G^{\prime}}(v)=2$, ապա $S(v,\beta)\in
\{\{1,2\},\{2,4\},\{4,5\}\}$,

$\bullet$ եթե $d_{G^{\prime}}(v)=1$, ապա $S(v,\beta)\in
\{\{1\},\{2\},\{4\},\{5\}\}$:

$G$ գրաֆի $\gamma$ ներկումը կստացվի $\beta$ ներկումից՝ $M$ կատարյալ զուգակցման կողերը $3$ գույնով ներկելով: Քանի որ $M$-ը ծածկում էր $G$-ի բոլոր գագաթները, $\gamma$ ներկման դեպքում գագաթների սպեկտրները կբավարարեն հետևյալ պայմաններին.

$\bullet$ եթե $d_{G^{\prime}}(v)=5$, ապա $S(v,\beta)=
[1,5]$,

$\bullet$ եթե $d_{G^{\prime}}(v)=4$, ապա
$S(v,\beta)\in \{[1,4],\{1,2,3,5\},\{1,3,4,5\},[2,5]\}$,

$\bullet$ եթե $d_{G^{\prime}}(v)=3$, ապա $S(v,\beta)\in
\{[1,3],[2,4],[3,5]\}$,

$\bullet$ եթե $d_{G^{\prime}}(v)=2$, ապա $S(v,\beta)\in
\{\{1,3\},\{2,3\},\{3,4\},\{3,5\}\}$,

$\bullet$ եթե $d_{G^{\prime}}(v)=1$, ապա $S(v,\beta) = \{3\}$:

Այսպիսով, $\gamma$-ն $G$-ի միջակայքային $(5,1)$-ներկում է և $w^1(G)=5$:
\end{proof}

\begin{corollary}
\label{c2_bipartite_Delta5_nopm} Եթե $G$-ն կատարյալ զուգակցում ունեցող երկկողմանի գրաֆ է, որի համար $\Delta(G)=5$, ապա $G \square K_2 \in \mathfrak{N}$:
\end{corollary}

\begin{theorem}
\label{t2_bipartite_Delta6_2factor} Եթե $G$-ն երկկողմանի գրաֆ է, որի համար $\Delta(G)=6$ և որն ունի 
$2$-համասեռ կմախքային ենթագրաֆ, ապա $G\in \mathfrak{N}^{1}$ և $w^{1}(G)=6$:
\end{theorem}
\begin{proof}[Ապացույց]
Դիցուք $G$-ն 6 առավելագույն աստիճանով երկկողմանի գրաֆ է:
Դիցուք $F$-ը $G$-ի $2$-ֆակտոր է: Դիտարկենք $G^{\prime}=G-E(F)$ գրաֆը:
Պարզ է, որ $G^{\prime}$-ը երկկողմանի գրաֆ է, որի համար
$\Delta(G^{\prime})=4$: Թեորեմ \ref{t2_bipartite_Delta4}-ի ապացույցի նման կարող ենք ցույց տալ, որ $G^{\prime}$-ը ունի $\alpha$ միջակայքային
$(4,1)$-ներկում այնպիսին, որ կամայական $v\in V(G^{\prime})$ գագաթի համար, որի համար $d_{G^{\prime}}(v)\in \{1,2,4\}$, $S(v,\alpha)$-ն ամբողջ թվերի միջակայք է: Այժմ կառուցենք $G^{\prime}$-ի նոր $\beta$ ներկումը վերցնելով $\alpha$ ներկման գույները և փոխարինելով 3 և 4 գույները համապատասխանաբար 5-ով և 6-ով: Պարզ է, որ ցանկացած $v\in
V(G^{\prime})$ գագաթի համար ունենք, որ

$\bullet$ եթե $d_{G^{\prime}}(v)=4$, ապա $S(v,\beta)=
\{1,2,5,6\}$,

$\bullet$ եթե $d_{G^{\prime}}(v)=3$, ապա
$S(v,\beta)\in \{\{1,2,5\},\{1,2,6\},\{1,5,6\},\{2,5,6\}\}$,

$\bullet$ եթե $d_{G^{\prime}}(v)=2$, ապա $S(v,\beta)\in
\{\{1,2\},\{2,5\},\{5,6\}\}$,

$\bullet$ եթե $d_{G^{\prime}}(v)=1$, ապա $S(v,\beta)\in
\{\{1\},\{2\},\{5\},\{6\}\}$:

Այժմ սահմանենք $G$-ի $\gamma$ ներկումը հետևյալ կերպ.

1) ցանկացած $e\in E(G^{\prime})$ կողի համար $\gamma(e)=\beta(e)$,

2) քանի որ $F$-ը $G$-ում իրենից ներկայացնում է զույգ ցիկլերի միավորում, ներկենք $F$-ի կողերը հերթականորեն 3 և 4 գույներով:

Քանի որ $G$-ն ունի 2-ֆակտոր, կամայական $v\in V(G)$ գագաթի համար ստանում ենք, որ

$\bullet$ եթե $d_{G}(v)=6$, ապա $S(v,\gamma)=[1,6]$,

$\bullet$ եթե $d_{G}(v)=5$, ապա
$S(v,\gamma)\in \{[1,5],[2,6],\{1,2,3,4,6\},\{1,3,4,5,6\}\}$,

$\bullet$ եթե $d_{G}(v)=4$, ապա $S(v,\gamma)\in
\{[1,4],[2,5],[3,6]\}$,

$\bullet$ եթե $d_{G}(v)=3$, ապա
$S(v,\gamma)\in \{[2,4],[3,5],\{1,3,4\},\{3,4,6\}\}$,

$\bullet$ եթե $d_{G}(v)=2$, ապա $S(v,\gamma)=\{3,4\}$:

Այստեղից հետևում է, որ $\gamma$-ն $G$-ի միջակայքային $(6,1)$-ներկում է:
\end{proof}

\begin{corollary}
\label{c2_bipartite_Delta6_2factor} Եթե $G$-ն երկկողմանի գրաֆ է, որի համար $\Delta(G)=6$ և ունի 2-ֆակտոր, ապա $G \square K_2 \in \mathfrak{N}$:
\end{corollary}
\end{hide}
