Կարգացուցակների կառուցման բազմաթիվ խնդիրներ բերվում են տարաբնույթ պայմաններով գրաֆների կողային ներկումների խնդիրների ուսումնասիրությանը: Մասնավորապես, «պատուհան» չունեցող դասացուցակների գոյության և կառուցման խնդիրները մոդելավորվում են գրաֆների (մուլտիգրաֆների) միջակայքային կողային ներկումների միջոցով: Այս աշխատանքը նվիրված է այդպիսի ներկումների և նրանց տարբեր բնույթի ընդհանրացումների հետազոտմանը: 

$G$ մուլտիգրաֆի գագաթների և կողերի բազմությունը նշանակենք, համապատասխանաբար, $V(G)$-ով և $E(G)$-ով: $\Delta(G)$-ով նշանակենք գրաֆի առավելագույն աստիճանը: $\alpha : E(G) \rightarrow \mathbb{N}$ ֆունկցիան կոչվում է $G$ մուլտիգրաֆի ճիշտ կողային ներկում, եթե ցանկացած $v\in V(G)$ գագաթին կից կողերը ներկված են զույգ առ զույգ տարբեր գույներով: Եթե $\alpha$-ն ճիշտ կողային ներկում է, $S(v,\alpha)$-ով նշանակում ենք $v$ գագաթին կից կողերի գույների բազմությունը, իսկ $\underline{S}(v, \alpha)$-ով և $\overline{S}(v, \alpha)$-ով՝ այդ բազմության փոքրագույն և մեծագույն գույները:
    
$\alpha : E(G) \rightarrow \{1,\ldots,t\}$ ճիշտ կողային ներկումը կոչվում է $G$ մուլտիգրաֆի միջակայքային $t$-ներկում, եթե կամայական $i$ թվի համար, $i=1,\ldots,t$, գոյություն ունի $e$ կող, որ $\alpha(e)=i$, իսկ կամայական $v \in V(G)$ գագաթի համար $S(v,\alpha)$ բազմությունը հանդիսանում է բնական թվերի միջակայք:

$\mathfrak{N}_t$-ով նշանակենք այն մուլտիգրաֆների բազմությունը, որոնք ունեն միջակայքային կողային $t$-ներկում, իսկ $\mathfrak{N}$-ով՝ $\mathfrak{N} = \bigcup\limits_{t \geq 1}{\mathfrak{N}_t}$, բոլոր միջակայքային կողային ներկելի գրաֆների բազմությունը: Եթե $G \in \mathfrak{N}$, $w(G)$-ով և $W(G)$-ով նշանակենք $t$-ի փոքրագույն և մեծագույն արժեքները, որոնց դեպքում $G$-ն ունի միջակայքային կողային $t$-ներկում:

Քանի որ ոչ բոլոր մուլտիգրաֆները ունեն միջակայքային ներկումներ, դիտարկվում են միջակայքային կողային ներկումների ընդհանրացումներ: Կամայական $G$ մուլտիգրաֆի $\alpha$ կողային ներկման համար սահմանվում է այդ ներկման դեֆիցիտը՝ $def(G, \alpha) = \sum\limits_{v\in V(G)}{\left(\overline{S}(v,\alpha)-\underline{S}(v,\alpha) - |S(v,\alpha)|+1\right)}$: $G$ մուլտիգրաֆի դեֆիցիտը՝ $def(G)$-ն, $G$-ի բոլոր ճիշտ կողային ներկումների դեֆիցիտներից նվազագույնն է: $w_{def}(G)$-ով և $W_{def}(G)$-ով նշանակում ենք $t$-ի փոքրագույն և մեծագույն արժեքները, որոնց համար $G$-ն ունի $t$ գույներով և $def(G)$ դեֆիցիտով ճիշտ կողային ներկում: 

$G$ և $H$ գրաֆների $G\square H$ դեկարտյան արտադրյալը սահմանվում է հետևյալ կերպ.
\begin{align*}
V(G \square H) &= V(G) \times V(H) \\
E(G \square H) &= \left\{(u_1,v_1)(u_2,v_2) : (u_1=u_2 \text{ և } v_1v_2 \in E(H)) \text{ կամ } (v_1=v_2 \text{ և } u_1u_2 \in E(G) \right\}
\end{align*}

Այս աշխատանքում դիտարկվել են մուլտիգրաֆների տարբեր դասերի՝ $\mathfrak{N}$ դասին պատկանելու խնդիրներ, այդ դասին պատկանող մուլտիգրաֆների համար՝ $w(G)$ և $W(G)$ պարամետրերի գնահատման և ճշգրիտ արժեքների որոշման խնդիրներ, միջակայքային ներկումներում մասնակցող գույների հնարավոր քանակի որոշման խնդիրներ, գրաֆների տարբեր գործողությունների նկատմամբ միջակայքային ներկելիության կայունության խնդիրներ, ինչպես նաև ուսումնասիրվել են միջակայքային ներկում չունեցող գրաֆների դեֆիցիտը և $w_{def}(G)$ ու $W_{def}(G)$ պարամետրերը:

Աշխատանքում ստացվել են հետևյալ արդյունքները:

1) Միջակայքային ներկումներ ունեցող գրաֆների և մուլտիգրաֆների $w(G)$ և $W(G)$ պարամետրերի, ինչպես նաև միջակայքային ներկում չունեցող գրաֆների $w_{def}(G)$ և $W_{def}(G)$ պարամետրերի համար տրվել են հասանելի գնահատականներ,

2) $K_{2n}$ լրիվ գրաֆների միջակայքային կողային ներկումների և այդ գրաֆների հատուկ տիպի ֆակտորիզացիաների համարժեքության հիման վրա ստացվել են  $W(K_{2n})$ պարամետրի նոր ստորին և վերին գնահատականներ, գտնվել են այդ պարամետրի ճշգրիտ արժեքները $n \leq 12$ արժեքների համար,

3) Ստացվել են լրիվ բազմակողմանի գրաֆների, արտաքին հարթ գրաֆների, ցանցերի, գլանների, տոռերի և Հեմինգի գրաֆների միջակայքային կողային ներկումների գոյության վերաբերյալ արդյունքներ, $w(G)$ և $W(G)$ պարամետրերի գնահատականներ, որոշ դեպքերում՝ նաև ճշգրիտ արժեքներ,

4) Ցույց է տրվել գրաֆների միջակայքային ներկելիության կապը այդ գրաֆների մասնակցությամբ դեկարտյան արտադրյալների միջակայքային ներկելիություն հետ, էապես ուժեղացվել են համասեռ գրաֆների մասնակցությամբ դեկարտյան արտադրյալների համար $W(G \square H)$ պարամետրի հայտնի գնահատականները, ապացուցվել է երկկողմանի գրաֆների մասնակցությամբ դեկարտյան արտադրյալների մի շարք դասերի միջակայքային ներկելիությունը,

5) Գտնվել են որոշ գրաֆների դեֆիցիտի ճշգրիտ արժեքները, հաստատվել է Բորովիցկա-Օլշեվսկայի, Դրգաշ-Բուրչարդտի և Հալուշչակի հիպոթեզը, ցույց է տրվել, որ արտաքին հարթ գրաֆները բավարարում են դեֆիցիտի մասին հիպոթեզին, համաձայն որի ցանկացած $G$ գրաֆի համար $def(G) \leq |V(G)|$,

6) Մասնակի լուծում է տրվել միջակայքային ներկում չունեցող փոքրագույն երկկողմանի գրաֆների մասին Ջենսեն-Տոֆտի խնդրին, կառուցվել են այդպիսի գրաֆների և մուլտիգրաֆների հայտնի փոքրագույն օրինակները, համակարգչային հաշվարկների միջոցով ցույց է տրվել, որ ոչ ավել, քան 16 գագաթ պարունակող բոլոր երկկողմանի գրաֆները ունեն միջակայքային ներկումներ: