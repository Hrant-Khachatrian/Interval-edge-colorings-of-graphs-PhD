
Աշխատանքում դիտարկվում են ոչ կողմնորոշված հասարակ գրաֆներ՝ առանց պատիկ կողերի և օղերի, ինչպես նաև մուլտիգրաֆներ, որտեղ թույլատրվում են պատիկ կողեր: $G$ գրաֆի (մուլտիգրաֆի) գագաթների և կողերի բազմությունները նշանակենք, համապատասխանաբար, $V(G)$-ով և $E(G)$-ով: Ցանկացած $v\in V(G)$-ի համար $d_G(v)$-ով նշանակենք այդ գագաթի \textit{աստիճանը} $G$-ում, $\delta(G)$-ով և $\Delta(G)$-ով նշանակենք գրաֆի (մուլտիգրաֆի) նվազագույն և առավելագույն աստիճանները: $e$ կողի պատիկությունը մուլտիգրաֆում նշանակում ենք $\mu(e)$-ով:

Ցանկացած $a$ և $b$ բնական թվերի համար ($a\leq b$) $[a,b]$-ով կնշանակենք $[a,b] = \{a,a+1,\ldots,b\}$ բազմությունը: $a$ և $b$ բնական թվերի ամենամեծ ընդհանուր բաժանարարը կնշանակենք $(a,b)$-ով:
 
Ցանկացած $u,v\in V(G)$ գագաթների համար $d(u,v)$-ով նշանակենք $u$ և $v$ գագաթների միջև հեռավորությունը $G$ գրաֆում (մուլտիգրաֆում): $v \in V(G)$ գագաթի համար սահմանենք $\epsilon(v)$ թիվը հետևյալ կերպ.
\begin{center}
$\epsilon(v) = \max_{u\in V(G)}{d(u,v)}$ ,
\end{center}
իսկ $G$ գրաֆի (մուլտիգրաֆի) \textit{տրամագիծը}՝ 
\begin{center}
$\mathrm{diam}(G) = \max_{v\in V(G)}{\epsilon(v)}$:
\end{center}

$G$ գրաֆում (մուլտիգրաֆում) ամենամեծ ցիկլի երկարությունը կոչվում է \textit{պարագիծ} և նշանակվում է $c(G)$-ով: Երկու կողերի միջև \textit{հեռավորությունը}՝ $d(e,e')$, սահմանվում է որպես այդ կողերի ծայրակետերի միջև հեռավորություններից նվազագույնը:

Ցիկլը կոչվում է \textit{էյլերյան}, եթե այն անցնում է մուլտիգրաֆի բոլոր գագաթներով և կողերով, ընդ որում՝ յուրաքանչյուր կողով ճիշտ մեկ անգամ: Էյլերյան ցիկլ ունեցող գրաֆը (մուլտիգրաֆը) կոչվում է \textit{էյլերյան գրաֆ} (\textit{էյլերյան մուլտիգրաֆ}): Ցիկլը կոչվում է \textit{համիլտոնյան}, եթե այն անցնում է գրաֆի յուրաքանչյուր գագաթով ճիշտ մեկ անգամ:

Գրաֆը կոչվում է \textit{հարթ}, եթե այն հնարավոր է պատկերել հարթության վրա այնպես, որ գրաֆի կողերը հատվեն միայն գագաթներում: Հարթության վրա պատկերված հարթ գրաֆը հարթությունը տրոհում է տիրույթների, որոնք կոչվում են \textit{նիստեր}: Հարթ գրաֆը կոչվում է \textit{արտաքին հարթ}, եթե այն հնարավոր է հարթության վրա պատկերել այնպես, որ նրա բոլոր գագաթները պատկանեն արտաքին (անվերջ) նիստին:

Կողերի $M$ բազմությունը կոչվում է \textit{զուգացկում}, եթե $M$-ի ցանկացած երկու կող հարևան չեն: Կասենք, որ $M$ զուգացկումը ծածկում է $v$ գագաթը, եթե $v$-ն կից է $M$-ի կողերից որևէ մեկին: $M$-ը կոչվում է \textit{կատարյալ զուգակցում}, եթե այն ծածկում է $G$ մուլտիգրաֆի բոլոր գագաթները: $\alpha'(G)$-ով կնշանակենք $G$ մուլտիգրաֆի ամենաշատ կողեր պարունակող զուգակցման հզորությունը: $\mathfrak{F} = \left\{F_1,\ldots,F_n \right\}$ կատարյալ զուգակցումների բազմությունը կանվանենք $G$ մուլտիգրաֆի \textit{$1$-ֆակտորիզացիա}, եթե $G$-ի կամայական կող պատկանում է $\mathfrak{F}$-ի զուգակցումներից ճիշտ մեկին:



\bigskip

$G$ և $H$ գրաֆների $G\square H$ \textit{դեկարտյան արտադրյալը} սահմանվում է հետևյալ կերպ.
Դիցուք $V(G)=\left\{u_1,\ldots,u_m\right\}$, իսկ $V(H)=\left\{v_1,\ldots,v_n\right\}$: Այդ դեպքում $G\square H$ դեկարտյան արտադրյալն է.
\begin{align*}
V(G \square H) &= \left\{ (u_i,v_j), 1\leq i\leq m, 1\leq j\leq n \right\} \\
E(G \square H) &= \bigcup\limits_{i=1}^{m}\left\{(u_i,x)(u_i,y) : xy \in E(H) \right\} \cup 
\bigcup\limits_{j=1}^{n}\left\{(x,v_j)(y,v_j) : xy \in E(G) \right\}
\end{align*}

Երբ $G$-ն և $H$-ը կապակցված գրաֆներ են, ապա $G\square H$-ն ևս կապակցված է: Ընդ որում, $\Delta(G\square H)=\Delta(G)+\Delta(H)$ և $\mathrm{diam}(G\square H)=\mathrm{diam}(G)+\mathrm{diam}(H)$: Պարզ է, որ ցանկացած $G$, $H$ և $F$ գրաֆների համար՝
\begin{center}
$G \square H \cong H \square G$ և $(G \square H) \square F \cong G \square (H \square F)$:
\end{center}



\bigskip

$\alpha : E(G) \rightarrow \mathbb{N}$ ֆունկցիան կոչվում է $G$ մուլտիգրաֆի \textit{ճիշտ կողային ներկում}, եթե $\forall v \in V(G)$ գագաթին կից կողերը ներկված են զույգ առ զույգ տարբեր գույներով:
Եթե $\alpha$ ճիշտ կողային ներկումը օգտագործում է միայն $1,\ldots,t$ գույները, ընդ որում՝ $\forall i$ թվի համար, $1 \leq i \leq t$, գոյություն ունի $e_i \in E(G)$ այնպիսին, որ $\alpha(e_i)=i$, ապա $\alpha$-ն կանվանենք $G$ մուլտիգրաֆի \textit{ճիշտ կողային $t$-ներկում}:

Եթե $\alpha$-ն $G$-ի ճիշտ կողային ներկում է, ապա կամայական $v$ գագաթի \textit{սպեկտրը}՝ $S(v,\alpha)$, այդ գագաթին կից կողերի գույների բազմությունն է: Պարզ է, որ երբ $\alpha$-ն ճիշտ ներկում է, $|S(v,\alpha)|=d_G(v)$ ցանկացած $v \in V(G)$ գագաթի համար: Սպեկտրի նվազագույն և առավելագույն թվերը կնշանակենք, համապատասխանաբար, $\underline{S}(v, \alpha)$-ով և $\overline{S}(v, \alpha)$-ով:

Դիցուք $V' \subseteq V(G)$ $G$ մուլտիգրաֆի գագաթների ենթաբազմություն է: $G[V']$-ով կնշանակենք $V'$ բազմության գագաթներով ծնված ենթագրաֆը: Մեզ պետք կգան նաև հետևյալ նշանակումները.
\begin{center}
	$S_\cup(V',\alpha) = \bigcup\limits_{v \in V'}{S(v,\alpha)}$ \\
	$S_\cap(V',\alpha) = \bigcap\limits_{v \in V'}{S(v,\alpha)}$
\end{center}

Ակնհայտ է, որ կամայական մուլտիգրաֆ ունի ճիշտ կողային ներկում (տարբեր կողեր կարող ենք ներկել տարբեր գույներով): Տրված $G$ մուլտիգրաֆի ճիշտ կողային ներկումներում անհրաժեշտ գույների նվազագույն քանակը կոչվում է \textit{քրոմատիկ դաս} և նշանակվում է $\chi'(G)$-ով: Ըստ Վիզինգի հայտնի թեորեմի՝
\begin{center}
$\Delta(G) \leq \chi'(G) \leq \Delta(G) + \mu(G)$,
\end{center}
որտեղ $\mu(G)$-ն $G$ գրաֆում կողերի առավելագույն պատիկությունն է \cite{Vizing1965}: Մասնավորապես, սովորական գրաֆների համար քրոմատիկ դասը կամ հավասար է առավելագույն աստիճանին, կամ առավելագույն աստիճանից մեծ է մեկով:

$\alpha$ ճիշտ կողային $t$-ներկումը կանվանենք \textit{միջակայքային կողային $t$-ներկում}, եթե ցանկացած $v \in V(G)$ գագաթին կից կողերը ներկված են $d_G(v)$ հաջորդական գույներով: Երբ $\alpha$-ն միջակայքային կողային $t$-ներկում է, բոլոր $v \in V(G)$ գագաթների համար $\overline{S}(v,\alpha)-\underline{S}(v,\alpha)=d_G(v)-1$:

Նկատենք, որ եթե $G$-ն կապակցված մուլտիգրաֆ է, ապա միջակայքային կողային ներկման սահմանման երկրորդ կետը կարելի է չստուգել:
\begin{lemma}
\label{t1_lemma} Եթե $\alpha$-ն $G$ կապակցված գրաֆի ճիշտ կողային $t$-ներկում է, այնպիսին որ ցանկացած $v\in V(G)$ գագաթին կից կողերը ներկված են հաջորդական գույներով, ընդ որում՝ $\min_{e\in E(G)}\{\alpha(e)\}=1$, $\max_{e\in
E(G)}\{\alpha(e)\}=t$, ապա $\alpha$-ն $G$-ի միջակայքային $t$-ներկում է:
\end{lemma}
\begin{proof}[Ապացույց] Լեմման ապացուցելու համար բավական է ցույց տալ, որ $G$-ի $\alpha$ ներկման մեջ օգտագործվում են բոլոր գույները:

Դիցուք $u$-ն և $w$-ն այնպիսի գագաթներ են, որ $1\in S(u,\alpha)$ և $t\in S(w,\alpha)$: Ենթադրենք $P=v_{1},\ldots,v_{k}$-ն (որտեղ $u=v_{1}$ և $v_{k}=w$) $u,w$-շղթան է $G$-ում: Երբ $k=1$, ապա $t\in S(u,\alpha)$ և բոլոր գույները $u$-ին կից կողերին են: Ենթադրենք $k\geq 2$. $v_{i}\in V(P)$ գագաթների համար $S(v_{i},\alpha)$ բազմությունները միջակայքեր են, ընդ որում ցանկացած $2\leq i\leq k$-ի համար $S(v_{i-1},\alpha)$ և $S(v_{i},\alpha)$ միջակայքերը ունեն ընդհանուր գույն: Այսպիսով, $S(v_{1},\alpha),\ldots,S(v_{k},\alpha)$ բազմությունները ծածկում են $[1,t]$ միջակայքը:
\end{proof}

Նշանակենք $\mathfrak{N}_t$-ով այն մուլտիգրաֆների բազմությունը, որոնք ունեն միջակայքային կողային $t$-ներկում, իսկ $\mathfrak{N}$-ով՝ $\mathfrak{N}=\bigcup\limits_{t\geq1}{\mathfrak{N}_t}$ բոլոր միջակայքային կողային ներկելի մուլտիգրաֆների բազմությունը:

Երբ $G \in \mathfrak{N}$, $G$-ի միջակայքային կողային ներկման մեջ օգտագործվող գույների նվազագույն և առավելագույն քանակները նշանակենք, համապատասխանաբար, $w(G)$-ով և $W(G)$-ով:

Ոչ բոլոր գրաֆներն ունեն միջակայքային ներկումներ: Ամենապարզ օրինակը $K_3$-ն է: Եթե $G$-ն միջակայքային ներկելի չէ, ապա կարելի է սահմանել չափեր, թե որքանով է $G$-ն «հեռու» միջակայքային ներկելի լինելուց: Այսպիսի չափերից ամենաշատը ուսումնասիրվել է գրաֆի \textit{դեֆիցիտը}, որը ներմուծվել է Գիառոյի, Կուբալի և Մալաֆիյսկու կողմից \cite{GiaroKubaleMalafiejski1999}:

Եթե $\alpha$-ն $G$-ի ճիշտ կողային ներկում է, ապա $v\in V(G)$ գագաթի դեֆիցիտը հետևյալ թիվն է. 
\begin{center}
$def(v, \alpha)=\overline{S}(v,\alpha) - \underline{S}(v,\alpha) - |S(v,\alpha)| + 1$:
\end{center}
$G$ մուլտիգրաֆի ճիշտ $\alpha$ ներկման դեֆիցիտը բոլոր գագաթների դեֆիցիտների գումարն է. 
$def(G, \alpha) = \sum\limits_{v \in V(G)}{def(v,\alpha)}$: 
$G$ մուլտիգրաֆի դեֆիցիտը $G$-ի բոլոր ճիշտ ներկումների դեֆիցիտներից փոքրագույնն է.
$def(G) = \min\limits_{\alpha}{def(G,\alpha)}$, որտեղ մինիմումը վերցվում է ըստ $G$-ի բոլոր ճիշտ ներկումների: Պարզ է, որ $def(G)$-ն կախված կողերի նվազագույն թիվն է, որն անհրաժեշտ է ավելացնել $G$-ին այն միջակայքային ներկելի դարձնելու համար: 

Կամայական $G$ մուլտիգրաֆի համար $w_{def}(G)$-ով և $W_{def}(G)$-ով կնշանակենք $t$-ի ամենափոքր և ամենամեծ արժեքները, որոնց համար $G$-ն ունի $\alpha$ ճիշտ կողային $t$-ներկում $def(G,\alpha)=def(G)$ դեֆիցիտով:

Մուլտիգրաֆի՝ միջակայքային ներկելիությունից «հեռավորության» մեկ այլ չափ է \textit{անցքերի թիվը}: Եթե $\alpha$-ն $G$-ի ճիշտ կողային ներկում է, ապա նրա անցքերի թիվը գրաֆի բոլոր գագաթների դեֆիցիտների մաքսիմումն է. $gn(G, \alpha) = \max\limits_{v \in V(G)}{def(v,\alpha)}$: $G$ մուլտիգրաֆի անցքերի թիվը $G$-ի բոլոր ճիշտ ներկումների անցքերի թվերից փոքրագույնն է. $gn(G) = \min\limits_{\alpha}{gn(G,\alpha)}$, որտեղ մինիմումը վերցվում է ըստ $G$-ի բոլոր ճիշտ կողային ներկումների:

Անցքերի թվի գաղափարը անմիջականորեն կապված է միջակայքային $(t,h)$-ներկումների հետ, որոնք ներմուծվել են \cite{PetrosyanArakelyan2007}-ում: $\alpha : E(G) \rightarrow \{1,\ldots,t\}$ ճիշտ կողային ներկումը կոչվում է $G$ մուլտիգրաֆի \textit{միջակայքային $(t,h)$-ներկում}, եթե բոլոր գույները օգտագործված են և կամայական $v \in V(G)$ գագաթի համար, $def(v, \alpha) \leq h$:

$\mathfrak{N}_t^h$-ով նշանակում ենք բոլոր այն գրաֆների բազմությունը, որոնք ունեն միջակայքային $(t,h)$-ներկումներ: $\mathfrak{N}^h = \bigcup_{t \geq 1}{\mathfrak{N}_t^h}$: Պարզ է, որ $G$ գրաֆի անցքերի թիվը $G$ այն նվազագույն $h$-ն է, որի համար  $G \in \mathfrak{N}^h$: Եթե $G \in \mathfrak{N}^h$, կասենք, որ $G$-ն \textit{միջակայքային $h$-անցք-ներկելի} է: 

Եթե $G \in \mathfrak{N}^h$, ապա $w^h(G)$-ով և $W^h(G)$-ով կնշանակենք $t$-ի ամենափոքր և ամենամեծ արժեքները, որոնց համար $G$-ն ունի միջակայքային $(t,h)$-ներկում: 

Հետևյալ անհավասարությունները անմիջապես բխում են սահմանումներից.
\begin{center}
$gn(G) \leq def(G) \leq gn(G)|V(G)|$:
\end{center}

Չսահմանված հասկացությունները և նշանակումները կարելի է գտնել \cite{AsratianDenleyHaggvist1998,Harary1969,Kubale2004,West1996,PMK2015}-ում: