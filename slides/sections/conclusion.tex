\begin{frame}{Հիմնական արդյունքներն ու հետևությունները}
\begin{enumerate}
\justifying
    \item Միջակայքային ներկումներ ունեցող գրաֆների և մուլտիգրաֆների $w(G)$ և $W(G)$ պարամետրերի, ինչպես նաև միջակայքային ներկում չունեցող գրաֆների $w_{def}(G)$ և $W_{def}(G)$ պարամետրերի համար տրվել են հասանելի գնահատականներ,
    
    \item $K_{2n}$ լրիվ գրաֆների միջակայքային կողային ներկումների և այդ գրաֆների հատուկ տիպի ֆակտորիզացիաների համարժեքության հիման վրա ստացվել են  $W(K_{2n})$ պարամետրի նոր ստորին և վերին գնահատականներ, գտնվել են այդ պարամետրի ճշգրիտ արժեքները $n \leq 12$ արժեքների համար,
    \seti
\end{enumerate}
\end{frame}

\begin{frame}{Հիմնական արդյունքներն ու հետևությունները}{(շարունակություն)}
\begin{enumerate}
    \justifying
    \conti
    \item Ստացվել են լրիվ բազմակողմանի գրաֆների, արտաքին հարթ գրաֆների, ցանցերի, գլանների, տոռերի և Հեմինգի գրաֆների միջակայքային կողային ներկումների գոյության վերաբերյալ արդյունքներ, $w(G)$ և $W(G)$ պարամետրերի գնահատականներ, որոշ դեպքերում՝ նաև ճշգրիտ արժեքներ,

    \item %Ցույց է տրվել գրաֆների միջակայքային ներկելիության կապը այդ գրաֆների մասնակցությամբ դեկարտյան արտադրյալների միջակայքային ներկելիություն հետ, էապես ո
    Ուժեղացվել են համասեռ գրաֆների մասնակցությամբ դեկարտյան արտադրյալների համար $W(G \square H)$ պարամետրի հայտնի գնահատականները, ապացուցվել է երկկողմանի գրաֆների մասնակցությամբ դեկարտյան արտադրյալների մի շարք դասերի միջակայքային ներկելիությունը,
    
    \seti
\end{enumerate}
\end{frame}

\begin{frame}{Հիմնական արդյունքներն ու հետևությունները}{(շարունակություն)}
\begin{enumerate}
    % \fontsize{10pt}{11}\selectfont
    \justifying
    \conti
    \item %Գտնվել են որոշ գրաֆների դեֆիցիտի ճշգրիտ արժեքները, 
    Հաստատվել է Բորովիցկա-Օլշեվսկայի, Դրգաշ-Բուրչարդտի և Հալուշչակի հիպոթեզը, ցույց է տրվել, որ արտաքին հարթ գրաֆները բավարարում են դեֆիցիտի մասին հիպոթեզին, համաձայն որի ցանկացած $G$ գրաֆի համար $def(G) \leq |V(G)|$,

    \item Մասնակի լուծում է տրվել միջակայքային ներկում չունեցող փոքրագույն երկկողմանի գրաֆների մասին Ջենսեն-Տոֆտի խնդրին, կառուցվել են այդպիսի գրաֆների և մուլտիգրաֆների հայտնի փոքրագույն օրինակները, համակարգչային հաշվարկների միջոցով ցույց է տրվել, որ ոչ ավել, քան 16 գագաթ պարունակող բոլոր երկկողմանի գրաֆները ունեն միջակայքային ներկումներ:
    
    \seti
\end{enumerate}
\end{frame}
